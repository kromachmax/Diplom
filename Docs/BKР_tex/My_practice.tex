%%%% Начало преамбулы %%%%
%%
%% Пожалуйста, не изменяйте файлы преамбулы
%%
\RequirePackage{etoolbox} % Для надежных определений команд
\newcommand*{\anyptfilebase}{template_settings/bpfont} 
\newcommand*{\anyptsize}{14pt} % Определяем размер шрифта как 14pt
\RequirePackage[l2tabu,orthodox]{nag} 
\documentclass[extrafontsizes,a4paper,\anyptsize,oneside,openany]{memoir}
\input{template_settings/common/setup}               
\input{template_settings/common/packages}  
\input{template_settings/Dissertation/dispackages}         
\input{template_settings/Dissertation/userpackages}         
\input{template_settings/Dissertation/setup}               
\input{template_settings/Dissertation/preamblenames}       
\input{template_settings/common/styles}    
\input{template_settings/Dissertation/disstyles}           
\input{template_settings/Dissertation/userstyles}          
\input{template_settings/biblio/bibliopreamble}
\input{template_settings/Dissertation/inclusioncontrol}
\input{template_settings/common/TO-DO-list}
%%
%%%% Конец преамбулы %%%% % лучше не редактировать / please, keep unmodified

\setcounter{docType}{2} % лучше не редактировать / please, keep unmodified

%%%% Настройки автора / Author settings
%% 
\input{my_folder/my_settings} % добавляем свои команды / update your commands


\begin{document} % начало документа
	
\input{template_settings/common/renames} % Заполнить сведения, 
										 % в т.ч. ключевые слова и аннотацию.

%%% Титульник отчета по практике / Practice report title 
%%
%% добавить лист в pdf-навигацию 
%% add to pdf navigation menu
%%
\pdfbookmark[-1]{\pdfTitle}{tit}
%%
\thispagestyle{empty}%
\makeatletter
\newgeometry{top=2cm,bottom=2cm,left=3cm,right=1cm,headsep=0cm,footskip=0cm}
\savegeometry{NoFoot}%
\makeatother


% TODO Exact match of font size
%{\centering%
%	\footnotesize
%	\MakeUppercase{Федеральное государственное автономное образовательное учреждение\\высшего образования}\\%
%		{\bfseries <<\SPbPU>>\\%
%			\institute\\%
%			\School}
%}

{\centering%
	\small%
	\MakeUppercase{\SPbPUOfficialPrefix}\\
	{\bfseries %2020 - указание на изменения, которые могут быть введены в 2020 году
	<<\MakeUppercase{\SPbPU}>>\\%
	\MakeUppercase{ФИЗИКО-МЕХАНИЧЕСКИЙ ИНСТИТУТ}\\
	\MakeUppercase{ВЫСШАЯ ШКОЛА ПРИКЛАДНОЙ МАТЕМАТИКИ И ВЫЧИСЛИТЕЛЬНОЙ ФИЗИКИ}
	}
\par}%
	

\vspace{0pt plus1fill} %число перед fill = кратность относительно некоторого расстояния fill, кусками которого заполнены пустые места


\noindent


%\vspace{0pt plus2fill} %


{\centering%
	{\bfseries{} 
	Отчет о прохождении преддипломной практики\\
	на тему: <<Децентрализованное решение задачи о назначениях целей группе роботов>>}\\

\intervalS\normalfont%

	\uline{Кромачев Максим Александрович , гр. 5030102/10201}%

\intervalS\normalfont%

\par}%

%\intervalS% %ОБЯЗАТЕЛЬНО ДОБАВИТЬ ОТСТУП, ЕСЛИ ХВАТАЕТ МЕСТА


{\noindent {\bfseries Направление подготовки:} {\expandafter \ulined 01.03.02 Прикладная математика и информатика}.}\par


{\noindent {\bfseries Место прохождения практики:} {\expandafter \ulined СПБПУ, ФизМех, ВШПМиВФ}.} % включая фактический адрес для практики в сторонней организации по договору


{\noindent {\bfseries Сроки практики:} \uline{с 02.02.25 по 28.05.25.}}\par


{\noindent {\bfseries Руководитель практики от \SPbPUOfficialShort:}} {\expandafter \ulined \SupervisorFull, \SupervisorJob, \SupervisorDegree.} %Ф.И.О., должность, степень



{\noindent \bfseries 
	Консультант практики от \SPbPUOfficialShort:
%	Консультанты практики от \SPbPUOfficialShort:
} 
{\noindent \expandafter \ulined \ConsultantExtraFull, \ConsultantExtraDegree}.%,      %% первый консультант Ф.И.О., должность, степень
%{\noindent \expandafter \ulined \ConsultantExtraTwoFull, \ConsultantExtraTwoDegree.}  %% второй консультант Ф.И.О., должность, степень 

{\noindent {\bfseries Оценка:} \uline{\hspace*{0.1\textheight}} % НЕ ЗАПОЛНЯЕМ!

\vspace{0pt plus1fill}%

\noindent
\begin{tabularx}{\linewidth}{lXl}
	Руководитель практики		&	&\\
	от \SPbPUOfficialShort		&	& \Supervisor     \\[\mfloatsep] % если не хватает места, закомментировать

	Консультант практики		&	&\\
	от \SPbPUOfficialShort		&	& \ConsultantExtra\\[\mfloatsep]
%							&	& \ConsultantExtraTwo\\[\mfloatsep]
	
	Обучающийся				&	&\Author\\[\mfloatsep]
	Дата: \uline{\PracticeEndDate}		&	&
\end{tabularx} %


%
\vspace{0pt plus4fill}% 

\restoregeometry
\newpage				 % Титульный лист
										 % Убираем footnotes, консультанта, если нет

\input{my_folder/contents}  	         % Оглавление


\chapter*{Введение}
\addcontentsline{toc}{chapter}{Введение}

Развитие робототехники и распределенных систем управления в последние десятилетия привело к значительному росту интереса к задачам оптимального распределения ресурсов в условиях ограниченной коммуникации. Такие задачи, известные как задачи назначения, требуют эффективного распределения ограниченного набора ресурсов (например, задач или объектов) между агентами (роботами) для максимизации общей выгоды. В условиях ограниченных областей связи, когда не все роботы могут взаимодействовать друг с другом или отсутствует единый центр управления, традиционные алгоритмы, такие как венгерский метод, сталкиваются с вычислительными трудностями или вообще неприменимы. Это подчеркивает необходимость разработки новых алгоритмов, способных учитывать топологические ограничения и обеспечивать высокую эффективность.

\textbf{Актуальность исследования} обусловлена растущей потребностью в автоматизации сложных систем управления в робототехнике, где ограниченные области связи создают дополнительные проблемы для координации роботов. По данным исследований \cite{bertsekas1990}, аукционные алгоритмы демонстрируют высокую эффективность для задач назначения в распределенных системах, однако их применение в робототехнике с учетом топологических ограничений остается недостаточно изученным. Это определяет актуальность данной работы, направленной на разработку и исследование новых подходов к решению задач назначения.

\textbf{Объект исследования} --- распределенные системы управления роботами, функционирующие в условиях ограниченной коммуникации.

\textbf{Предмет исследования} --- алгоритмы назначения, обеспечивающие оптимальное распределение целей между роботами с учетом ограниченных областей связи.

\textbf{Цель исследования} --- повышение эффективности распределения целей в распределенных системах роботов с ограниченными областями связи путем разработки и применения модифицированного аукционного алгоритма.

\textbf{Задачи исследования}:
\begin{enumerate}
    \item Изучить существующие алгоритмы решения задачи назначения, включая аукционный и венгерский методы.
    \item Проанализировать влияние ограниченных областей связи на эффективность алгоритмов назначения.
    \item Разработать модификацию аукционного алгоритма, учитывающую топологические ограничения в системах роботов.
    \item Реализовать предложенный алгоритм и венгерский алгоритм в программной среде.
    \item Провести сравнительное тестирование алгоритмов на модельных задачах с различными характеристиками коммуникационных ограничений.
\end{enumerate}

\textbf{Теоретическая база исследования} включает работы по задачам назначения и групповому управлению роботами. Основой послужили исследования Д. Бертсекаса \cite{bertsekas1990}, описывающие аукционный алгоритм и его преимущества, а также работы Х. Куна \cite{kuhn1955} по венгерскому методу. Значительное внимание уделено работам по распределённым системам управления \cite{pshikhopov2015},  \cite{kalyaev2009}, а также проблемам оптимального распределения целей в мультироботных системах \cite{gerkey2003}. В процессе подготовки исследования были изучены такие дисциплины, как «Алгоритмы и структуры данных», «Робототехника» и «Теория оптимизации».

\textbf{Методологическая база} включает общенаучные методы (анализ, моделирование, эксперимент) и конкретно-научные методы (методы линейного программирования, венгерский алгоритм, аукционные алгоритмы, алгоритмы на графах). В работе применены подходы к распараллеливанию
 вычислений, а также методы сравнительного анализа алгоритмов.

\textbf{Информационная база} включает материалы учебных дисциплин, данные из научных публикаций, а также результаты моделирования, полученные в ходе выполнения данной ВКР.

\textbf{Степень научной разработанности} проблемы характеризуется значительным вниманием к задачам назначения в работах Д. Бертсекаса, Х. Куна и других авторов \cite{bertsekas1990}-\cite{gerkey2003}. Исследования группового управления роботами, включая работы В.Х. Пшихопова \cite{pshikhopov2015} и И.А. Каляева \cite{kalyaev2009}, подчеркивают важность учета коммуникационных ограничений. Однако адаптация аукционных алгоритмов к ограниченным областям связи в робототехнике остается малоисследованной, что определяет необходимость разработки новых подходов.

\textbf{Научная новизна} заключается в разработке модифицированного аукционного алгоритма, адаптированного для распределенных систем роботов с ограниченными областями связи. Новизна проявляется в учете топологических ограничений коммуникации, что позволяет повысить эффективность алгоритма по сравнению с традиционными методами. Также новизна состоит в сравнительном анализе аукционного и венгерского алгоритмов в контексте робототехнических приложений.

\textbf{Практическая значимость} заключается в возможности применения разработанного алгоритма для оптимизации распределения задач в системах управления роботами, работающих в условиях ограниченной коммуникации, таких как складская логистика, автономные транспортные системы и роевые робототехнические платформы. Результаты работы могут быть использованы для повышения эффективности реальных систем и дальнейших исследований в области распределенного управления.

\textbf{Апробация результатов} включает представление результатов на защите ВКР и размещение работы на портале СПбПУ.

Введение определяет рамки исследования, обосновывая актуальность проблемы оптимального распределения целей в распределённых системах управления роботами, особенно в условиях ограниченной коммуникации. Оно формулирует цель работы — повышение эффективности распределения задач путём разработки нового подхода, основанного на аукционном алгоритме, а также связывает поставленные задачи с главами работы, задавая структуру исследования.

Первая глава посвящена анализу существующих алгоритмов назначения. В ней рассматриваются классические методы, их математические основы, преимущества и ограничения, с акцентом на их применимость в робототехнических системах. Особое внимание уделяется анализу их применимости в условиях, где связи между роботами ограничены, что подчёркивает необходимость нового подхода.

Вторая глава сосредоточена на разработке модифицированного аукционного алгоритма. Она описывает математическую модель задачи, учитывающую топологические ограничения сети, и представляет изменения, внесённые в классический аукционный метод, чтобы адаптировать его к условиям ограниченной связи. Также обсуждаются теоретические аспекты, включая сходимость и оценку производительности алгоритма.

Третья глава посвящена программной реализации предложенного модифицированного алгоритма и его классического аналога. В ней описывается используемая программная среда, структура кода и подходы к моделированию распределённых систем. Особое внимание уделяется методам тестирования и визуализации результатов, которые позволяют оценить поведение алгоритмов при различных сценариях.

Четвёртая глава фокусируется на тестировании и сравнительном анализе разработанного алгоритма с классическим методом. Она представляет результаты вычислительных экспериментов, демонстрирующих влияние ключевых параметров на производительность и точность. Графики и численные данные иллюстрируют, как алгоритм справляется с задачами в условиях ограниченной коммуникации.

Пятая глава обсуждает перспективы дальнейшего развития исследования. В ней рассматриваются возможные улучшения алгоритма. Также анализируются потенциальные области применения, подчёркивая практическую значимость разработки.

Заключение подводит итоги исследования, суммируя достигнутые результаты. Оно также указывает на направления для будущих исследований, которые могут расширить применимость разработанного подхода.
	    	 % Введение

% Начало основной части
\chapter{Анализ существующих алгоритмов назначения}
\label{ch:analysis}

\section{Введение в предметную область}
Задача назначения --- одна из ключевых проблем комбинаторной оптимизации, широко применяемая в робототехнике, логистике и управлении ресурсами. Согласно \cite{bertsekas1990}, задача заключается в распределении $n$ агентов (например, роботов) по $n$ объектам (например, задачам) для максимизации суммарной выгоды, заданной матрицей $\{a_{ij}\}$, где $a_{ij}$ --- выгода от назначения агента $i$ объекту $j$. Математически цель задачи:

\[
\sum_{i=1}^n a_{i j_i} \to \max,
\]

где $j_i$ --- объект, назначенный агенту $i$, и все $j_i$ различны.

В робототехнике задача назначения актуальна для распределения задач между роботами в условиях ограниченной коммуникации, когда роботы обмениваются информацией только с соседями. Такие ограничения, обусловленные топологией сети, требуют алгоритмов, эффективных в распределенных системах. Примеры приложений: складская логистика, автономные транспортные системы, роевые платформы \cite{kalyaev2009, gerkey2003}.

Цель главы --- проанализировать аукционный и венгерский алгоритмы, включая их математические основы, сходимость и оптимальность, и оценить их применимость к робототехнике. Анализ обосновывает необходимость модифицированного аукционного алгоритма для ограниченной коммуникации.

\section{Аукционный алгоритм}
Аукционный алгоритм, предложенный Д. Бертсекасом \cite{bertsekas1990}, представляет собой итеративный метод для решения задачи назначения, моделирующий процесс аукциона, где агенты делают ставки на объекты, а цены корректируются для достижения оптимального распределения.

\subsection{Постановка задачи}
Задача назначения (Assignment Problem) заключается в нахождении оптимального соответствия между $n$ агентами и $n$ объектами с учетом матрицы выгод $\{a_{ij}\}$, где $a_{ij}$ — выгода агента $i$ при назначении ему объекта $j$. Цель — максимизировать суммарную выгоду:
\[
\max \sum_{i=1}^n a_{i j_i},
\]
где $j_i$ — объект, назначенный агенту $i$, при условии, что каждые агент и объект участвуют ровно в одном назначении (биективное соответствие). Эта задача эквивалентна задаче линейного программирования с ограничениями:
\[
\begin{cases}
\sum_{i=1}^n x_{ij} = 1, & \forall j = 1, \ldots, n, \\
\sum_{j=1}^n x_{ij} = 1, & \forall i = 1, \ldots, n, \\
x_{ij} \in \{0, 1\}, & \forall i, j,
\end{cases}
\]
где $x_{ij} = 1$, если агент $i$ назначен объекту $j$, и $x_{ij} = 0$ в противном случае. Аукционный алгоритм решает эту задачу приближенно с заданной точностью $\varepsilon > 0$, достигая почти оптимального решения через итеративную корректировку цен.

\subsection{Описание алгоритма}
Согласно \cite{bertsekas1990}, алгоритм работает с матрицей выгод $\{a_{ij}\}$ и начинается с произвольного распределения и цен $\{p_j\}$. Ключевые определения:
\begin{itemize}
    \item \textit{Почти счастье}: Агент $i$, назначенный объекту $j_i$, почти счастлив, если:
    \[
    a_{i j_i} - p_{j_i} \geq \max_{j=1,\ldots,n} \{a_{ij} - p_j\} - \varepsilon,
    \]
    где $\varepsilon > 0$ — параметр точности.
    \item \textit{Почти равновесие}: Распределение и цены, при которых все агенты почти счастливы.
\end{itemize}

Шаги алгоритма \cite{bertsekas1990}:
\begin{enumerate}
    \item Проверить, все ли агенты почти счастливы. Если да, завершить.
    \item Выбрать агента $i$, не почти счастливого, и найти объект $j_i$:
    \[
    j_i \in \arg \max_{j=1,\ldots,n} \{a_{ij} - p_j\}.
    \]
    \item Переназначить: агент $i$ получает $j_i$, а агент, ранее назначенный на $j_i$, получает объект, принадлежавший $i$.
    \item Увеличить цену $p_{j_i}$ на:
    \[
    \gamma_i = v_i - w_i + \varepsilon,
    \]
    где $v_i = \max_j \{a_{ij} - p_j\}$, $w_i = \max_{j \neq j_i} \{a_{ij} - p_j\}$.
    \item Повторить с шага 1.
\end{enumerate}

\subsection{Сходимость алгоритма}
\begin{theorem}[Сходимость аукционного алгоритма \cite{bertsekas1990}]
\label{thm:auction_convergence}
Алгоритм завершается за конечное число шагов с распределением и ценами, почти в равновесии.
\end{theorem}

\textbf{Доказательство}. Если объект $j$ получает ставку, агент $i$, назначенный на $j$, становится почти счастливым, так как $j$ максимизирует $a_{ij} - p_j$. После увеличения $p_j$ на $\gamma_i = v_i - w_i + \varepsilon$:

\[
a_{i j} - (p_j + \gamma_i) = a_{i j} - p_j - (v_i - w_i + \varepsilon) = w_i - \varepsilon \leq \max_{k \neq j} \{a_{ik} - p_k\} - \varepsilon,
\]

что удовлетворяет условию почти счастья. Агент $i$ остается почти счастливым, пока удерживает $j$, так как цены $p_k$ ($k \neq j$) не уменьшаются, а $a_{i j} - p_j$ может только уменьшаться.

Агенты, не почти счастливые, назначены на объекты без ставок. Если все объекты получат хотя бы одну ставку, все агенты станут почти счастливы, и алгоритм завершится. Предположим, некоторые объекты никогда не получают ставок. Объект $j$ с $m$ ставками имеет цену $p_j \geq m \varepsilon$. При большом $m$ $a_{i j} - p_j$ становится меньше $a_{i k} - p_k$ для объекта $k$ без ставок ($p_k = 0$), и агент выберет $k$. Таким образом, все объекты получат ставку, или алгоритм завершится ранее, когда все агенты почти счастливы. Конечность шагов следует из конечности числа объектов.

\subsection{Оценка итераций}
\begin{claim}[Оценка итераций \cite{bertsekas1990}]
\label{claim:auction_iterations}
Количество итераций пропорционально $C / \varepsilon$, где $C = \max_{i,j} |a_{ij}|$.
\end{claim}

\textbf{Доказательство}. Каждая ставка увеличивает цену объекта минимум на $\varepsilon$. Максимальная выгода $C$ ограничивает рост цен. В худшем случае цены достигают порядка $C$, и число ставок (итераций) составляет $O(C / \varepsilon)$.

\subsection{Оптимальность}
\begin{theorem}[Оптимальность аукционного алгоритма \cite{bertsekas1990}]
\label{thm:auction_optimality}
Если алгоритм завершается с почти равновесным распределением, суммарная выгода находится в пределах $n \varepsilon$ от оптимальной. При целочисленных $a_{ij}$ и $\varepsilon < 1/n$ распределение оптимально.
\end{theorem}

\textbf{Доказательство}. Оптимальная выгода (примитивная задача):

\[
A^* = \max_{\{k_i\}} \sum_{i=1}^n a_{i k_i}, \quad k_i \neq k_m \text{ для } l \neq m.
\]

Двойственная задача:

\[
D^* = \min_{p_j} \left\{ \sum_{j=1}^n p_j + \sum_{i=1}^n \max_j \{a_{ij} - p_j\} \right\}.
\]

Для любого распределения $\{ (i, k_i) \}$ и цен $\{ p_j \}$:

\[
\sum_{i=1}^n a_{i k_i} \leq \sum_{j=1}^n p_j + \sum_{i=1}^n \max_j \{a_{ij} - p_j\},
\]

так как $\max_j \{a_{ij} - p_j\} \geq a_{i k_i} - p_{k_i}$. Следовательно, $A^* \leq D^*$.

При почти равновесии для распределения $\{ (i, j_i) \}$:

\[
a_{i j_i} - p_{j_i} \geq \max_j \{a_{ij} - p_j\} - \varepsilon.
\]

Суммируем по $i$:

\[
\sum_{i=1}^n (a_{i j_i} - p_{j_i}) \geq \sum_{i=1}^n \max_j \{a_{ij} - p_j\} - n \varepsilon.
\]

Добавим $\sum_{j=1}^n p_j$:

\[
\sum_{i=1}^n a_{i j_i} \geq \sum_{j=1}^n p_j + \sum_{i=1}^n \max_j \{a_{ij} - p_j\} - n \varepsilon.
\]

Поскольку $D^* \leq \sum_{j=1}^n p_j + \sum_{i=1}^n \max_j \{a_{ij} - p_j\}$, то:

\[
\sum_{i=1}^n a_{i j_i} \geq D^* - n \varepsilon \geq A^* - n \varepsilon.
\]

Так как $\sum_{i=1}^n a_{i j_i} \leq A^*$, получаем:

\[
A^* - n \varepsilon \leq \sum_{i=1}^n a_{i j_i} \leq A^*.
\]

Для целочисленных $a_{ij}$ и $\varepsilon < 1/n$, $A^* - \sum_{i=1}^n a_{i j_i} < 1$, и, так как разность целочисленная, $A^* = \sum_{i=1}^n a_{i j_i}$.

% Describing properties of the auction algorithm with updated details
\subsection{Свойства, преимущества и недостатки}
Свойства \cite{bertsekas1990}:
\begin{itemize}
    \item Сходимость за конечное число шагов: алгоритм завершается, как только все объекты получают хотя бы одну ставку или все люди становятся почти счастливыми (теорема \ref{thm:auction_convergence}).
    \item Оптимальность в пределах $n \varepsilon$: итоговая выгода назначения находится в пределах $n \varepsilon$ от оптимального значения, а при $\varepsilon < 1/n$ и целых $a_{ij}$ назначение оптимально (теорема \ref{thm:auction_optimality}).
    \item Производительность зависит от начальных цен: цены, близкие к оптимальным, значительно сокращают число раундов, особенно при использовании $\varepsilon$-масштабирования.
\end{itemize}

% Outlining advantages with refined insights
Преимущества:
\begin{itemize}
    \item Интуитивная экономическая модель: алгоритм имитирует реальный аукцион, упрощая понимание и интерпретацию.
    \item Гибкость для адаптации: легко модифицируется для асимметричных задач, транспортных проблем и задач минимальной стоимости потока \cite{bertsekas1990}.
    \item Хорошо подходит для параллельных и распределенных систем: допускает асинхронные и параллельные реализации, эффективен для разреженных задач \cite{gerkey2003}.
\end{itemize}

% Detailing disadvantages with precise considerations
Недостатки:
\begin{itemize}
    \item Число итераций зависит от $C / \varepsilon$: для больших $C = \max_{i,j} |a_{ij}|$ и малого $\varepsilon$ требуется больше раундов, примерно пропорционально $C / \varepsilon$.
    \item Чувствительность к выбору $\varepsilon$: большое $\varepsilon$ снижает точность (итоговая выгода дальше от оптимальной), малое $\varepsilon$ увеличивает число итераций.
    \item Чувствительность к начальным ценам: плохие начальные цены замедляют сходимость, хотя $\varepsilon$-масштабирование смягчает эту проблему.
\end{itemize}


\section{Венгерский алгоритм}
Венгерский алгоритм, разработанный Х. Куном \cite{kuhn1955}, решает задачу назначения через редукцию матрицы выгод или эквивалентную задачу максимального паросочетания в двудольном графе \cite{emaxx2025}.

\subsection{Описание через матрицу}
Алгоритм работает с матрицей выгод $\{a_{ij}\}$ размера $n \times n$ \cite{kuhn1955}:
\begin{enumerate}
    \item Для каждого ряда вычесть минимальный элемент: $a_{ij} \gets a_{ij} - \min_j a_{ij}$.
    \item Для каждого столбца вычесть минимальный элемент: $a_{ij} \gets a_{ij} - \min_i a_{ij}$.
    \item Найти минимальное число строк и столбцов, покрывающих все нули.
    \item Если число линий равно $n$, найти назначение (нули соответствуют оптимальным парам) и завершить.
    \item Иначе найти минимальный непокрытый элемент $\delta$, вычесть $\delta$ из непокрытых элементов, прибавить к элементам на пересечении линий, вернуться к шагу 3.
\end{enumerate}

\subsection{Графовый подход}
Как указано в \cite{emaxx2025}, задача назначения эквивалентна нахождению максимального паросочетания в двудольном графе $G = (V_1 \cup V_2, E)$, где $V_1$ --- агенты, $V_2$ --- объекты, а ребра $(i,j)$ имеют вес $a_{ij}$. Алгоритм:
\begin{enumerate}
    \item Построить матрицу $\{a_{ij}\}$ и редуцировать ее, как выше.
    \item Сформировать граф, где ребра соответствуют нулям в матрице.
    \item Найти максимальное паросочетание (например, с помощью алгоритма Куна или Форда-Фалкерсона).
    \item Если паросочетание покрывает $n$ вершин, оно оптимально. Иначе корректировать матрицу ($\delta$) и обновить граф.
\end{enumerate}

\subsection{Доказательство оптимальности}
Алгоритм основан на двойственности линейного программирования \cite{kuhn1955}. Примитивная задача:

\[
\max \sum_{i,j} a_{ij} x_{ij}, \quad \sum_{j} x_{ij} = 1, \sum_{i} x_{ij} = 1, x_{ij} \geq 0.
\]

Двойственная задача:

\[
\min \sum_{i} u_i + \sum_{j} v_j, \quad u_i + v_j \geq a_{ij},
\]

где $u_i$, $v_j$ --- двойственные переменные. Редукция матрицы и корректировка $\delta$ обеспечивают выполнение условий $u_i + v_j \geq a_{ij}$, а равенство $u_i + v_j = a_{ij}$ для выбранных пар дает оптимальность.

\subsection{Свойства, преимущества и недостатки}
Свойства \cite{kuhn1955, emaxx2025}:
\begin{itemize}
    \item Сходимость за конечное число шагов.
    \item Оптимальность: всегда находит максимальную выгоду.
    \item Сложность: $O(n^3)$ для матричной реализации.
\end{itemize}

Преимущества:
\begin{itemize}
    \item Гарантированная оптимальность.
    \item Простота реализации \cite{emaxx2025}.
    \item Поддержка в библиотеках.
\end{itemize}

Недостатки:
\begin{itemize}
    \item Сложность $O(n^3)$ для больших $n$.
    \item Требует полной матрицы $\{a_{ij}\}$, что проблематично при ограниченной коммуникации \cite{gerkey2003}.
    \item Нет параллелизма \cite{bertsekas1989}.
\end{itemize}

\section{Применение алгоритмов в робототехнике}
В робототехнике задача назначения используется для распределения задач между роботами в условиях ограниченной коммуникации \cite{kalyaev2009, pshikhopov2015}. Аукционный алгоритм \cite{bertsekas1990} поддерживает параллелизм, что подходит для роевых систем, но не учитывает топологические ограничения \cite{gerkey2003}. Венгерский алгоритм \cite{kuhn1955, emaxx2025} эффективен при полном доступе к данным, но неприменим без центрального управления \cite{kalyaev2009}. Исследования \cite{pshikhopov2015, gerkey2003} подчеркивают необходимость адаптации алгоритмов для динамических сетей.

\section{Постановка проблемы}
Аукционный алгоритм \cite{bertsekas1990} не учитывает ограничения связи, снижая эффективность в распределенных системах \cite{gerkey2003}. Венгерский алгоритм \cite{kuhn1955, emaxx2025} требует полной матрицы выгод, что неосуществимо без централизации \cite{kalyaev2009}. Необходим модифицированный аукционный алгоритм, который:
\begin{itemize}
    \item Учитывает ограниченные области связи.
    \item Сохраняет параллелизм.
    \item Обеспечивает эффективность и близость к оптимальности.
\end{itemize}

Цель --- разработать такой алгоритм и сравнить с венгерским методом.

\section{Выводы}
Рассмотрены аукционный \cite{bertsekas1990} и венгерский \cite{kuhn1955, emaxx2025} алгоритмы. Аукционный алгоритм поддерживает параллелизм, но требует адаптации для ограниченной коммуникации. Венгерский алгоритм оптимален, но непригоден для распределенных систем. Необходим модифицированный аукционный алгоритм, учитывающий топологические ограничения.

	         	 % Глава 1
\ContinueChapterBegin % размещать главы <<подряд>> 
\chapter{Разработка модифицированного аукционного алгоритма}
\label{ch2}

В данной главе представлена разработка модифицированного аукционного алгоритма для распределенных систем роботов с ограниченными областями связи. На основе анализа аукционного алгоритма Бертсекаса \cite{bertsekas1990} предлагается модификация, учитывающая топологические ограничения сети, при которых роботы обмениваются информацией только с соседями, объединенными в группы на основе радиуса связи. Описывается математическая модель задачи, а также алгоритмические изменения, обеспечивающие независимую обработку связных компонент и устойчивость к отсутствию коммуникации между группами. Формулируются теоремы об эффективности и близости решения к оптимальному.

\section{Математическая модель задачи}

Задача назначения формулируется следующим образом. Дано множество из \( n \) роботов \( R = \{r_1, r_2, \ldots, r_n\} \) и множество из \( m \) целей \( T = \{t_1, t_2, \ldots, t_m\} \). Каждый робот \( r_i \) имеет координаты \( p_i = (x_i, y_i) \), а каждая цель \( t_j \) --- координаты \( q_j = (X_j, Y_j) \). Затраты на назначение робота \( r_i \) цели \( t_j \) заданы матрицей \( \{c_{ij}\} \), где \( c_{ij} \) представляет время, необходимое роботу \( i \) для достижения цели \( j \), вычисляемое по формуле:

\[
c_{ij} = \frac{d(p_i, q_j)}{v_i},
\]

\noindent где \( d(p_i, q_j) = \sqrt{(X_j - x_i)^2 + (Y_j - y_i)^2} \) --- евклидово расстояние, \( v_i \) --- скорость робота \( i \), предполагаемая постоянной и одинаковой для всех роботов (например, \( v_i = 1 \)). Для решения задачи в терминах максимизации матрица затрат \( c_{ij} \) преобразуется в матрицу выгод \( \alpha_{ij} \):

\[
\alpha_{ij} = C_{\text{max}} - c_{ij}, \quad \text{где} \quad C_{\text{max}} = \max_{i,j} c_{ij}.
\]

Все роботы имеют доступ ко всем целям, поэтому \( c_{ij} \) и \( \alpha_{ij} \) определены для любых \( i \) и \( j \).

Матрица связи роботов \( V = \{v_{ij}\} \) определяет топологию сети: \( v_{ij} = 1 \), если \( d(p_i, p_j) \leq R \) (роботы \( r_i \) и \( r_j \) могут обмениваться информацией), и \( v_{ij} = 0 \) в противном случае. Роботы объединяются в группы (связные компоненты графа \( V \)), внутри которых проводится аукцион.

Цель задачи --- максимизировать суммарную выгоду:

\[
\sum_{i=1}^n \alpha_{i j_i} \to \max,
\]

где \( j_i \) --- цель, назначенная роботу \( r_i \), что эквивалентно минимизации суммарного времени:

\[
\sum_{i=1}^n c_{i j_i} \to \min,
\]

поскольку \( \sum_{i=1}^n \alpha_{i j_i} = \sum_{i=1}^n (C_{\text{max}} - c_{i j_i}) = n C_{\text{max}} - \sum_{i=1}^n c_{i j_i} \), и максимизация \( \sum \alpha_{i j_i} \) соответствует минимизации \( \sum c_{i j_i} \). Условия задачи:

\begin{itemize}
    \item Цель может быть назначена нескольким роботам в процессе аукциона, но в итоговом решении для каждой цели \( t_j \) учитывается только робот с максимальной выгодой \( \alpha_{ij} \), что эквивалентно минимальному времени \( c_{ij} \). Остальные назначения для этой цели отбрасываются.
    \item Роботы обмениваются информацией о ценах и назначениях только внутри связных компонент графа связи \( V \).
\end{itemize}

\section{Модификация аукционного алгоритма}

Модифицированный аукционный алгоритм адаптирует подход Бертсекаса \cite{bertsekas1990} для учета отсутствия коммуникации между роботами. Ключевые изменения включают:

\begin{itemize}
    \item Полная видимость целей: Все цели доступны каждому роботу, поэтому \( c_{ij} \) и \( \alpha_{ij} \) определены для всех \( i, j \).
    \item Разделение на связные компоненты: Роботы группируются по графу связи \( V \), что позволяет проводить аукцион независимо для каждой связанной компоненты.
    \item Независимая обработка компонент: Каждая связная компонента обрабатывается отдельно, моделируя распределенную систему без коммуникации между группами.
    \item Разрешение конфликтов: Если несколько роботов из разных компонент выбирают одну цель, учитывается только робот с максимальной выгодой \( \alpha_{ij} \), что соответствует минимальному времени \( c_{ij} \), а остальные назначения отбрасываются.
\end{itemize}

\subsection{Описание алгоритма}

Алгоритм состоит из следующих шагов:

\begin{enumerate}
    \item \textbf{Разбиение на компоненты}:
    \begin{itemize}
        \item Каждый робот \( r_i \) определяет своих соседей, с которыми он может обмениваться информацией: \( v_{ik} = 1 \), если \( d(p_i, p_k) \leq R \), и \( v_{ik} = 0 \) иначе. Эта информация хранится локально.
        \item Каждый робот \( r_i \) участвует в распределённом обмене информацией со своими соседями (где \( v_{ik} = 1 \)), отправляя свой идентификатор и получая идентификаторы соседей. Этот процесс повторяется, пока роботы не определят состав своей связной компоненты \( C_l \).
        \item В результате каждый робот знает, к какой компоненте \( C_l \) он принадлежит, и список роботов в этой компоненте.
    \end{itemize}
    \item \textbf{Инициализация}:
    \begin{itemize}
        \item Каждый робот \( r_i \) знает свои координаты \( p_i = (x_i, y_i) \) и координаты всех целей \( \{q_j\} \). Робот вычисляет локальную строку матрицы затрат \( \{c_{ij}\} \), где \( c_{ij} = \frac{d(p_i, q_j)}{v_i}\).
        \item Через локальный обмен информацией с соседями (где \( v_{ik} = 1 \)) роботы в компоненте \( C_l \) совместно определяют \( C_{\text{max}} = \max_{i \in C_l, j} c_{ij} \).
        \item Каждый робот \( r_i \in C_l \) преобразует свою строку затрат в строку матрицы выгод: \( \alpha_{ij} = C_{\text{max}} - c_{ij} \).
        \item Каждый робот инициализирует цены целей \( \{p_j = 0\} \) и своё назначение \( j_i = -1 \) (не назначен).
    \end{itemize}
    \item \textbf{Аукцион в компонентах}:
    \begin{itemize}
        \item Для каждой компоненты \( C_l \):
        \begin{enumerate}
            \item Для каждого робота \( r_i \in C_l \) вычислить текущую прибыль: \( \alpha_{i j_i} - p_{j_i} \), если \( j_i \neq -1 \), иначе \( 0 \).
            \item Проверить условие почти счастья: \( \alpha_{i j_i} - p_{j_i} \geq \max_j \{\alpha_{ij} - p_j\} - \varepsilon \), где \( \varepsilon > 0 \).
            \item Если робот не почти счастлив, найти цель \( t_{j_i} \): \( j_i = \arg \max_j \{\alpha_{ij} - p_j\} \), включая фиктивную цель (\( \alpha_{i,-1} = 0 \)).
            \item Вычислить \( v_i = \max_j \{\alpha_{ij} - p_j\} \), \( w_i = \max_{j \neq j_i} \{\alpha_{ij} - p_j\} \).
            \item Если \( j_i = -1 \), переназначить робота на фиктивную цель без изменения цен.
            \item Иначе переназначить: робот \( r_i \) получает цель \( t_{j_i} \), прежний владелец \( t_{j_i} \) (если есть) становится неназначенным.
            \item Увеличить цену: \( p_{j_i} += v_i - w_i + \varepsilon \).
            \item Повторять, пока все роботы в \( C_l \) не станут почти счастливы.
        \end{enumerate}
    \end{itemize}
\end{enumerate}

\subsection{Сходимость алгоритма}

\begin{theorem}[Сходимость модифицированного аукционного алгоритма]
\label{thm:mod_auction_convergence}
Модифицированный аукционный алгоритм завершается за конечное число шагов, при котором все роботы в каждой связной компоненте удовлетворяют условию почти счастья, определенному в главе \ref{ch:analysis}.
\end{theorem}

\textbf{Доказательство}. 
Модифицированный алгоритм проводит аукцион в каждой связной компоненте \( C_l \), следуя шагам оригинального аукционного алгоритма, описанного в главе \ref{ch:analysis}. Согласно теореме \ref{thm:auction_convergence} (глава \ref{ch:analysis}), оригинальный аукционный алгоритм завершается за конечное число шагов, обеспечивая почти равновесие, при котором все назначенные роботы почти счастливы, то есть для каждого робота \( i \), назначенного цели \( j_i \), выполняется:
\[
\alpha_{i j_i} - p_{j_i} \geq \max_{j=1,\ldots,m} \{\alpha_{ij} - p_j\} - \varepsilon.
\]
Поскольку аукцион в каждой компоненте \( C_l \) идентичен оригинальному, сходимость внутри \( C_l \) гарантируется той же теоремой. Таким образом, алгоритм завершается за конечное число шагов, обеспечивая почти равновесие в каждой компоненте.

\subsection{Оценка итераций модифицированного аукционного алгоритма}

\begin{theorem}[Оценка числа итераций модифицированного аукционного алгоритма]
\label{thm:mod_auction_iterations}
Пусть \( k \) — число связных компонент графа связи \( V \), \( n_l \) — число роботов в компоненте \( C_l \), \( m \) — число целей, \( C = \max_{i,j} |a_{ij}| \) — максимальная абсолютная величина выгоды от назначения робота \( i \) на цель \( j \), а \( \varepsilon \) — минимальный шаг увеличения цены цели в алгоритме. Тогда общее число итераций модифицированного аукционного алгоритма в худшем случае не превышает \(m \cdot \sum_{l=1}^k \frac{C}{\varepsilon}\).
\end{theorem}

\begin{proof}
Для каждой связной компоненты \( C_l \) графа связи \( V \) применяется оригинальный аукционный алгоритм. Согласно теореме \ref{thm:auction_iterations}, число итераций для компоненты \( C_l \) в худшем случае не превышает \( \frac{m \cdot C}{\varepsilon} \). Суммируя по всем \( k \) компонентам, получаем общее число итераций: \(m \cdot \sum_{l=1}^k \frac{C}{\varepsilon}\).
\end{proof}
\subsection{Оптимальность}

\begin{theorem}[Оптимальность модифицированного алгоритма]
\label{thm:mod_auction_optimality}
В каждой связной компоненте \( C_l \) модифицированный аукционный алгоритм при целочисленных выгодах \( \alpha_{ij} \) и \( \varepsilon < 1/n \) дает оптимальное решение локальной задачи аукциона, как определено в главе \ref{ch:analysis}. Однако общее решение после объединения компонент не гарантирует глобальной оптимальности из-за отсутствия коммуникации между компонентами.
\end{theorem}

\textbf{Доказательство}. 
В каждой связной компоненте \( C_l \) аукцион проводится по правилам оригинального аукционного алгоритма, описанного в главе \ref{ch:analysis}. Согласно теореме \ref{thm:auction_optimality} (глава \ref{ch:analysis}), при целочисленных \( \alpha_{ij} \) и \( \varepsilon < 1/n \) оригинальный аукционный алгоритм обеспечивает оптимальное решение, где суммарная выгода \( \sum_{i \in C_l} \alpha_{i j_i} \) равна локальному оптимуму. Таким образом, в каждой компоненте \( C_l \) достигается оптимальное локальное назначение. 

Однако глобальная оптимальность не гарантируется. Из-за отсутствия коммуникации между компонентами роботы в \( C_l \) не учитывают выгоды роботов из других компонент, что может привести к выбору локально оптимальных, но глобально неоптимальных назначений. Следовательно, суммарная выгода \( \sum_{i=1}^n \alpha_{i j_i} \) может быть меньше глобального оптимума.


\section{Выводы}

Разработан модифицированный аукционный алгоритм, учитывающий отсутствие коммуникации между роботами в распределенных системах, за исключением обмена внутри связных компонент. Алгоритм группирует роботов в связные компоненты с помощью матрицы связей \( V \), проводит аукцион независимо в каждой компоненте и обеспечивает полную видимость целей. Доказана сходимость за конечное число шагов (теорема \ref{thm:mod_auction_convergence}) и оптимальность в каждой компоненте для целочисленных выгод \( \alpha_{ij} \) при \( \varepsilon < 1/n \) (теорема \ref{thm:mod_auction_optimality}), но глобальная оптимальность не достигается из-за отсутствия коммуникации между компонентами. Преимущества включают простоту реализации, снижение коммуникационных затрат и адаптивность к реальным сценариям робототехники. Сформулированные гипотезы задают направления для дальнейших исследований эффективности и устойчивости алгоритма.	         	 % Глава 2
\chapter{Реализация и программное обеспечение} \label{ch3}

Глава посвящена реализации модифицированного аукционного и венгерского алгоритмов для сравнительного анализа. Описывается программная среда, выбранная для реализации, включая язык программирования и инструменты моделирования распределенных систем роботов. Приводится структура программного обеспечения, обеспечивающего тестирование алгоритмов.

\section{Программная среда}

Реализация выполнена на языке C++ с использованием библиотек Qt для визуализации данных. Генерация случайных данных реализована с использованием стандартных средств C++. Код совместим с платформами Windows и Unix.

\section{Структура реализации}

Программное обеспечение включает модули для аукционного и венгерского алгоритмов, а также логику тестирования и визуализации результатов. Аукционный алгоритм учитывает ограничения связи, разделяя роботов на группы и назначая задачи с учетом настраиваемого параметра радиуса связи, влияющего на скорость и точность. Венгерский алгоритм решает задачу оптимального назначения. Тестирование проводится на случайных данных с варьированием размеров задачи (5- 300 роботов/целей), радиусов связи (1-300 условных единиц) и параметра точности $\varepsilon$ (от $10^{-6}$ до 10.0).

\section{Тестирование и визуализация}

Тестирование включало сравнение времени выполнения, числа итераций и точности алгоритмов. Результаты сравнения представлены на графиках, что позволило оценить то, как настройка параметров точности $\varepsilon$ и радиуса связи R влияют на работу методов.           	 % Глава 3
\chapter{Экспериментальное исследование и анализ результатов} \label{ch4}

В четвертой главе представлены результаты экспериментального исследования модифицированного аукционного алгоритма в сравнении с венгерским алгоритмом. Описываются модельные задачи, имитирующие распределенные системы роботов с различной топологией сети связи. Приводятся метрики оценки (время выполнения, точность решения, устойчивость к изменениям сети). Анализируется эффективность предложенного алгоритма, подтверждаются гипотезы, сформулированные в главе 2. Результаты апробированы на научной конференции и через портал СПбПУ.           	 % Глава 4
\chapter{Экспериментальное исследование и анализ результатов} \label{ch5}

Глава посвящена направлениям дальнейшего развития модифицированного аукционного алгоритма и его применению в распределенных системах роботов. Рассматриваются возможные улучшения алгоритма, расширение модельных задач  и перспективы практической реализации. Обсуждаются потенциальные области применения и планы апробации результатов.           	 % Глава 5
\ContinueChapterEnd % завершить размещение глав <<подряд>>
% % %% Завершение основной части

\chapter*{Заключение} \label{ch-conclusion}
\addcontentsline{toc}{chapter}{Заключение}	% в оглавление 


Проведенное исследование демонстрирует, что разработанный модифицированный аукционный алгоритм является эффективным решением для задач оптимального распределения целей в распределенных системах управления роботами, функционирующих в условиях ограниченных областей связи. В отличие от классического венгерского алгоритма, который требует полной матрицы выгод и централизованного управления, что делает его непригодным для децентрализованных систем, модифицированный аукционный алгоритм учитывает топологические ограничения сети, позволяя роботам обмениваться информацией только внутри связных компонент. Это обеспечивает его применимость в реальных сценариях, где коммуникация между роботами ограничена радиусом связи.

Экспериментальные результаты, представленные в главе \ref{ch4}, подтверждают высокую эффективность алгоритма: он достигает высокой точности при разумных значениях радиуса связи и параметра $\varepsilon$, сохраняя приемлемое время выполнения. Алгоритм требует меньшего числа арифметических и логических операций по сравнению с венгерским, что делает его предпочтительным при высоких скоростях обмена данными ($F_{\text{обм}}$), несмотря на большее число итераций.

Модифицированный аукционный алгоритм обладает значительным потенциалом для применения в таких областях, как складская логистика, автономные транспортные системы и роевые робототехнические платформы. Однако отсутствие глобальной оптимальности из-за ограниченной коммуникации между компонентами указывает на необходимость дальнейших исследований, включая разработку механизмов ограниченного обмена информацией и адаптивной настройки параметра $\varepsilon$. Перспективы, описанные в главе  \ref{ch5}, подчеркивают возможность интеграции алгоритма в динамические сети и повышения его робастности к погрешностям измерений, что делает его универсальным инструментом для современных робототехнических систем.
        	 % Заключение

% % %% Наличие следующих перечней не исключает расшифровку сокращения и условного обозначения при первом упоминании в тексте!
% % \input{my_folder/acronyms}		         % Необязательная рубрика! Список сокращений и условных обозначений

% % % \input{my_folder/dictionary}    		 % Необязательная рубрика! Словарь терминов
% % % % По порядку после Списка сокращений и условных обозначений, если есть.	


\clearpage                                  % В том числе гарантирует, что список литературы в оглавлении будет с правильным номером страницы
%\hypersetup{ urlcolor=black }               % Ссылки делаем чёрными
%\providecommand*{\BibDash}{}                % В стилях ugost2008 отключаем использование тире как разделителя 
\urlstyle{rm}                               % ссылки URL обычным шрифтом
\ifdefmacro{\microtypesetup}{\microtypesetup{protrusion=false}}{} % не рекомендуется применять пакет микротипографики к автоматически генерируемому списку литературы
%\newcommand{\fullbibtitle}{Список литературы} % (ГОСТ Р 7.0.11-2011, 4)
%\insertbibliofull  
%\noindent
%\begin{group}
\chapter*{Список литературы}	
\label{references}
\addcontentsline{toc}{chapter}{Список использованных источников}	% в оглавление 

% Добавляем фиктивные цитаты в нужном порядке
\nocite{bertsekas1990, bertsekas1989, kalyaev2009, pshikhopov2015, kuhn1955, gerkey2003, emaxx2025}

\printbibliography[env=SSTfirst,sorting=none]  % Отключаем сортировку, используем заданный порядок
%\ifdefmacro{\microtypesetup}{\microtypesetup{protrusion=true}}{}
%\urlstyle{tt}                               % возвращаем установки шрифта ссылок URL
%\hypersetup{ urlcolor={urlcolor} }          % Восстанавливаем цвет ссылок		     % Список литературы

% % % Здесь можно поместить список иллюстративного материала

% % \appendix % не редактировать / keep unmodified


% % % \input{my_folder/appendix1}			     % Приложение 1

% % % \input{my_folder/appendix2}			 	 % Приложение 2


\end{document} % конец документа
