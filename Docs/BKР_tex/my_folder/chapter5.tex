\chapter{Перспективы дальнейшего развития и исследования}
\label{ch5}

% Вводное описание целей главы
Данная глава посвящена анализу перспектив дальнейшего развития модифицированного аукционного алгоритма, представленного в главе \ref{ch2}, и его применению в распределённых системах управления роботами. Рассматриваются возможные улучшения алгоритма, новые метрики оценки эффективности, а также потенциальные области практического применения.

\section{Поддержка гибкого распределения роботов по целям}
Текущая версия алгоритма предполагает, что каждая цель назначается только одному роботу(глава \ref{ch2}). Однако в общем случае может потребоваться более гибкое распределение роботов по целям. Это включает возможность назначения нескольких роботов на одну цель, а также распределение одного робота на несколько целей в зависимости от характеристик задачи, таких как приоритет цели, её сложность, временные ограничения, ресурсы роботов или требования к надёжности. Например, важная цель может требовать совместной работы нескольких роботов для ускорения выполнения, тогда как менее приоритетная цель может быть разделена между роботами с учётом их текущей загрузки. Перспективным направлением является модификация алгоритма, позволяющая задавать для каждой цели $t_j$ максимальное количество роботов $k_j$, а также учитывать возможность назначения одного робота на несколько целей. Все последующие направления исследований должны учитывать возможность гибкого распределения роботов по целям, чтобы обеспечить совместимость с этим подходом.

% Направление 1: Глобальная координация
\section{Обеспечение глобальной оптимальности}
Модифицированный аукционный алгоритм, описанный в главе \ref{ch2}, обеспечивает локальную оптимальность внутри связных компонент графа связи, однако не гарантирует глобальную оптимальность из-за отсутствия коммуникации между компонентами (теорема \ref{thm:mod_auction_optimality}). Для устранения этого ограничения перспективным направлением является разработка механизмов ограниченного обмена информацией между компонентами. Например, можно реализовать иерархическую структуру, где представители каждой компоненты обмениваются агрегированными данными о ценах и назначениях, или использовать периодическую синхронизацию для координации решений. Такие подходы позволят приблизить суммарную выгоду к глобальному оптимуму, сохраняя преимущества децентрализованного управления.

% Направление 2: Адаптивный выбор параметра ε
\section{Адаптивная настройка параметра $\varepsilon$}
Экспериментальные результаты (глава \ref{ch4}) показали, что параметр $\varepsilon$ существенно влияет на баланс между точностью и числом итераций алгоритма. Малые значения $\varepsilon$ (например, $\varepsilon \leqslant 10^{-2}$) обеспечивают высокую точность, но увеличивают вычислительные затраты, тогда как большие $\varepsilon$ сокращают время выполнения за счёт снижения качества назначений. Перспективным направлением является разработка адаптивных методов выбора $\varepsilon$ в зависимости от характеристик задачи, таких как размер задачи $n$, радиус связи $R$ и дисперсия ошибок измерений $\sigma_c$. 

% Направление 3: Динамическая топология сети
\section{Учёт динамической топологии сети}
Текущая версия алгоритма предполагает статическую топологию сети, где матрица связи $V$ фиксирована на протяжении выполнения аукциона. Однако в реальных системах роботы могут перемещаться, изменяя топологию связных компонент. Перспективным направлением является адаптация алгоритма к динамическим сетям путём интеграции методов обнаружения и обновления связных компонент в реальном времени. Это потребует разработки распределённых алгоритмов для отслеживания изменений в графе $V$ и динамической перегруппировки роботов, что повысит устойчивость алгоритма к изменениям.

% Направление 4: Учёт погрешностей
\section{Робастность к погрешностям измерений}
Как отмечено в главе \ref{ch4}, погрешности измерений расстояний $d(p_i, q_j)$ влияют на точность матрицы затрат $c_{ij}$ и, следовательно, матрицы выгод $\alpha_{ij}$. Для повышения надёжности алгоритма в реальных условиях перспективно интегрировать методы робастной оптимизации, учитывающие вероятностные распределения ошибок (например, $\varepsilon_{ij} \sim N(0, \sigma_c^2)$). Это может включать использование интервальных оценок для $c_{ij}$ или байесовских подходов для корректировки назначений с учётом неопределённости. Такие улучшения сделают алгоритм применимым в системах с ограниченной точностью сенсоров.


% Направление 7: Новые метрики оценки
\section{Разработка новых метрик оценки}
Для более полной оценки эффективности алгоритма перспективно ввести дополнительные метрики, такие как:
\begin{itemize}
    \item Устойчивость к изменениям топологии сети, измеряемая числом итераций, необходимых для адаптации к новым условиям.
    \item Адаптивность к изменениям числа целей, характеризующая скорость и точность перераспределения задач при добавлении или удалении целей в реальном времени.
    \item Равномерность распределения задач, оценивающая, насколько равномерно задачи распределяются между роботами.
\end{itemize}
Эти метрики позволят лучше сравнивать алгоритм с альтернативными подходами и оптимизировать его для конкретных приложений.

% Практические области применения
\section{Области практического применения}
Разработанный алгоритм имеет значительный потенциал для применения в различных областях робототехники и автоматизации. Основные направления включают:
\begin{itemize}
	\item \textbf{Военные робототехнические системы}: Координация роев дронов для выполнения боевых задач, таких как разведка или подавление вражеских объектов, в условиях ограниченной связи.
	\item \textbf{Поисково-спасательные операции}: Управление группами роботов или дронов для поиска пострадавших в зонах стихийных бедствий, таких как землетрясения или наводнения, с учетом нестабильной коммуникационной среды.
	\item \textbf{Промышленная автоматизация}: Распределение задач в децентрализованных производственных системах, где роботы работают в условиях ограниченного обмена данными.
\end{itemize}

% Заключение главы
\section{Выводы}
Перспективы дальнейшего развития модифицированного аукционного алгоритма включают улучшение глобальной оптимальности, адаптивную настройку параметров, учёт динамической топологии и погрешностей измерений. Введение новых метрик оценки работы алгоритма позволят лучше исследовать его универсальность и эффективность. Практическая значимость подтверждается применимостью алгоритма в военных робототехнических и поисково-спасательных системах, а также промышленной автоматизации.