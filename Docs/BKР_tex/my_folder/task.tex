%%%% Начало оформления заголовка - оставить без изменений !!! %%%%
\input{my_folder/task_settings}	% настройки - начало 
	
				{%\normalfont %2020
						\MakeUppercase{\SPbPU}}\\
				Физико-механический институт

\par}\intervalS% завершает input

				\noindent
				\begin{minipage}{\linewidth}
				\vspace{\mfloatsep} % интервал 	
				\begin{tabularx}{\linewidth}{Xl}
					&УТВЕРЖДАЮ      \\
					&Руководитель образователь-\\&ной программы «Прикладная\\& математика и информатика»     \\			
					&\underline{\hspace*{0.1\textheight}} К.Н. Козлов     \\
					&<<\underline{\hspace*{0.05\textheight}}>> \underline{\hspace*{0.1\textheight}}2025 г.  \\  
				\end{tabularx}
				\vspace{\mfloatsep} % интервал 	
				\end{minipage}

\intervalS{\centering\bfseries%

				ЗАДАНИЕ\\
				на выполнение %с 2020 года 
				%по выполнению % до 2020 года
				выпускной квалификационной работы


\intervalS\normalfont%

				студенту \uline{Кромачеву Максиму Александровичу{} гр.~5030102/10201}


\par}\intervalS%
%%%%
%%%% Конец оформления заголовка  %%%%
 	
	
	
\begin{enumerate}[1.]
	\item Тема работы: {\expandafter \ulined «Децентрализованное решение задачи о назначениях целей группе роботов».}
	%\item Тема работы (на английском языке): \uline{\thesisTitleEn.} % вероятно после 2021 года
	\item Срок сдачи студентом законченной работы: \uline{июнь 2025г.} 
	\item Исходные данные по работе: \uline{ТЗ на ОКР «БЭС-О», инструментальные средства: языки программирования С++, Python, пакет MATLAB, ключевые источники литературы \cite{bertsekas1990}, \cite{kuhn1955}, \cite{kalyaev2009}, \cite{gerkey2003}, \cite{pshikhopov2015}.}%
	\printbibliographyTask % печать списка источников % КОММЕНТИРУЕМ ЕСЛИ НЕ ИСПОЛЬЗУЕТСЯ
	% В СЛУЧАЕ, ЕСЛИ НЕ ИСПОЛЬЗУЕТСЯ МОЖНО ТАКЖЕ ЗАЙТИ В setup.tex и закомментировать \vspace{-0.28\curtextsize}
	\item Содержание работы (перечень подлежащих разработке вопросов):
	\begin{enumerate}[label=\theenumi\arabic*.]
		\item Введение. Обоснование актуальности проблемы.
		\item Постановка и формализация задачи.
		\item Обзор существующих подходов.
		\item Модификация алгоритма аукционного торга для случая ограниченных областей видимости и связи роботов. 
		\item Вычислительные эксперименты. 
		\item Результаты и их сравнительный анализ. 
		\item Выводы. 
		\item Заключение. 
	\end{enumerate}	
		\item Дата выдачи задания: \uline{03.02.2025.}
\end{enumerate}

\intervalS%можно удалить пробел

Руководитель ВКР \uline{\hspace*{0.1\textheight} И.Е. Ануфриев}


\intervalS%можно удалить пробел

Консультант \uline{\hspace*{0.1\textheight}С.А. Васильковский}


\intervalS%можно удалить пробел

%Консультант по нормоконтролю \uline{\hspace*{0.1\textheight} \ConsultantNorm}%ПОКА НЕ ТРЕБУЕТСЯ, Т.К. ОН У ВСЕХ ПО УМОЛЧАНИЮ

Задание принял к исполнению \uline{03.02.2025.}

\intervalS%можно удалить пробел

Студент \uline{\hspace*{0.1\textheight}  М.А. Кромачев}



\input{my_folder/task_settings_restore}	% настройки - конец