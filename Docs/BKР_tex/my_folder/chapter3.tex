\chapter{Реализация и программное обеспечение} \label{ch3}

Глава посвящена реализации модифицированного аукционного и венгерского алгоритмов для сравнительного анализа. Описывается программная среда, выбранная для реализации, включая язык программирования и инструменты моделирования распределенных систем роботов. Приводится структура программного обеспечения, обеспечивающего тестирование алгоритмов.

\section{Программная среда}

Реализация выполнена на языке C++ с использованием библиотек Qt для визуализации данных. Генерация случайных данных реализована с использованием стандартных средств C++. Код совместим с платформами Windows и Unix.

\section{Структура реализации}

Программное обеспечение включает модули для аукционного и венгерского алгоритмов, а также логику тестирования и визуализации результатов. Аукционный алгоритм учитывает ограничения связи, разделяя роботов на группы и назначая задачи с учетом настраиваемого параметра радиуса связи, влияющего на скорость и точность. Венгерский алгоритм решает задачу оптимального назначения. Тестирование проводится на случайных данных с варьированием размеров задачи (5- 300 роботов/целей), радиусов связи (1-300 условных единиц) и параметра точности $\varepsilon$ (от $10^{-6}$ до 10.0).

\section{Тестирование и визуализация}

Тестирование включало сравнение времени выполнения, числа итераций и точности алгоритмов. Результаты сравнения представлены на графиках, что позволило оценить то, как настройка параметров точности $\varepsilon$ и радиуса связи R влияют на работу методов.