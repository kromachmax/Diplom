\chapter{Экспериментальное исследование и анализ результатов}
\label{ch4}

\vspace{0.5cm}

В данной главе представлены результаты экспериментального исследования модифицированного аукционного алгоритма, описанного в главе \ref{ch2}, в сравнении с классическим венгерским алгоритмом. Рассматриваются модельные задачи, имитирующие распределенные системы роботов с различной топологией сети и радиусом связи \( r_v \). Описываются оценки, включающие количество операций, скорость сходимости, относительную точность. Анализируется эффективность предложенного алгоритма. Результаты визуализированы в виде графиков.

\section{Постановка эксперимента}

\subsection{Модельные задачи}

\vspace{0.3cm}

Для исследования эффективности модифицированного аукционного алгоритма были разработаны модельные задачи, имитирующие распределенные системы роботов. Параметры задач включали:

\begin{itemize}
    \item \textbf{Размер задачи}: Число роботов \( n \) и задач \( m \) варьировалось от 5 до 100, при этом \( n = m \) для сбалансированной задачи.
    \item \textbf{Радиус связи}: Рассматривались значения \( r_v \in \{10, 20, 30\} \), а также диапазон от 5 до 100 для анализа влияния радиуса на точность.
    \item \textbf{Координаты}: Координаты роботов \( \{p_i = (x_i, y_i)\} \) и задач \( \{q_j = (x_j, y_j)\} \) генерировались случайным образом в диапазоне \( [0, 100] \) с использованием генератора псевдослучайных чисел.
    \item \textbf{Матрица выгод}: Выгода \( \alpha_{ij} \) определялась как:
    \[
    \alpha_{ij} = \frac{D}{d(p_i, q_j) + \delta},
    \]
    где \( d(p_i, q_j) \) --- евклидово расстояние, D --- максимально возможное расстояние, \( \delta \) --- константа для предотвращения деления на ноль.
    \item \textbf{Матрица видимости}: Матрица \( V = \{v_{ij}\} \) формировалась так, что \( v_{ij} = 1 \), если \( d(p_i, p_j) \leq r_v \), и \( v_{ij} = 0 \) в противном случае.
    \item \textbf{Параметр \( \varepsilon \)}: Для аукционного алгоритма исследовались значения \( \varepsilon \) от \( 10^{-5} \) до 10 с логарифмическим шагом.
\end{itemize}

Каждая задача генерировалась 100 раз для усреднения результатов, что обеспечивало статистическую надежность.

\subsection{Методика проведения экспериментов}

Эксперименты проводились с использованием программной реализации на языке C++ с применением стандартных библиотек для обработки данных и построения графиков. Исследовались следующие аспекты:

\begin{enumerate}
    \item Сравнение количества элементарных арифметических и логических операций, а также обменов между роботами в аукционном и венгерском алгоритмах для различных размеров задачи и радиусов связи.
    \item Оценка числа итераций для фиксированного радиуса \( r_v = 200 \).
    \item Анализ относительной точности аукционного алгоритма для фиксированных размеров задачи \( n \in \{10, 30, 50\} \) и различных радиусов.
    \item Исследование влияния параметра \( \varepsilon \) на время выполнения и точность при \( n = 80 \) и \( r_v \in \{10, 20, 30\} \).
    \item Оценка точности аукционного алгоритма на каждой итерации для \( n \in \{10, 30, 50, 100\} \) и \( r_v = 200 \).
\end{enumerate}

Результаты представлены в виде графиков, сохраненных в формате PNG.

\section{Результаты экспериментов}

\subsection{Число операций и обменов}

\vspace{0.3cm}

Эксперименты проводились для \( n = m \) от 5 до 100 и радиуса \( r_v = 200 \). Результаты представлены на графике (см. рисунок \ref{fig:opeartion_chart}).

\begin{figure}[h]
    \centering
    \includegraphics[width=0.8\textwidth]{my_folder/images/operation_chart.png}
    \caption{Зависимость числа операций и обменов аукционного и венгерского алгоритмов от размера задачи.}
    \label{fig:opeartion_chart}
\end{figure}

\vspace{0.3cm}

Эксперименты показывают, что в аукционном алгоритме число арифметических и логических операций значительно меньше, чем в венгерском алгоритме, что делает его более эффективным с точки зрения вычислительной сложности для этих типов операций. Однако число обменов между роботами, которое характеризует количество итераций в аукционном алгоритме, значительно больше, что может замедлять выполнение, особенно при большом числе роботов и целей.

Для повышения эффективности аукционного алгоритма по сравнению с венгерским необходимо, чтобы роботы были оснащены технологиями быстрого обмена данными, обеспечивающими минимальное время передачи информации между собой. Это позволит сократить общее время выполнения алгоритма, сохраняя его преимущества в меньшем числе арифметических и логических операций.

Введение двух характеристик — \( F_{\text{выч}} \) (количество операций в секунду с вещественными или целыми числами) и \( F_{\text{обм}} \) (скорость обмена данными в системе связи, бит/сек) — позволяет оценить, какой подход окажется лучше в зависимости от их значений. Аукционный алгоритм будет предпочтительнее, если \( F_{\text{обм}} \) достаточно высоко, что компенсирует большое число обменов, а \( F_{\text{выч}} \) обеспечивает достаточную вычислительную мощность для обработки меньшего числа операций. Венгерский алгоритм, напротив, будет эффективнее при высоком \( F_{\text{выч}} \) и низком значении \( F_{\text{обм}} \), когда обмены не играют значительной роли из-за централизованного характера алгоритма.



\subsection{Относительная точность}
Точность аукционного алгоритма исследовалась для \( n \in \{10, 30, 50\} \) и \( r_v \) от 5 до 100 (см. рисунок \ref{fig:accuracy_chart}).

\begin{figure}[h]
    \centering
    \includegraphics[width=0.8\textwidth]{my_folder/images/accuracy_chart.png}
    \caption{Зависимость относительной точности аукционного алгоритма от радиуса видимости для различных размеров задачи.}
    \label{fig:accuracy_chart}
\end{figure}

Выводы:
\begin{itemize}
    \item Точность возрастает с увеличением \( r_v \), достигая 95--100\% при \( r_v \geq 50 \) для малых \( n \).
    \item Для больших \( n \) точность возрастает быстрее из-за меньшего числа связных компонент.
\end{itemize}

\subsection{Влияние параметра \( \varepsilon \)}

\vspace{0.3cm}

Влияние \( \varepsilon \) на время и точность исследовалось для \( n = 80 \) и \( r_v \in \{10, 20, 30\} \) (см. рисунки \ref{fig:epsilon_time_chart} и \ref{fig:epsilon_accuracy_chart}).

\begin{figure}[h]
    \centering
    \includegraphics[width=0.8\textwidth]{my_folder/images/epsilon_time_chart.png}
    \caption{Зависимость времени выполнения аукционного алгоритма от \( \varepsilon \) для различных радиусов видимости.}
    \label{fig:epsilon_time_chart}
\end{figure}

\begin{figure}[h]
    \centering
    \includegraphics[width=0.8\textwidth]{my_folder/images/epsilon_accuracy_chart.png}
    \caption{Зависимость относительной точности аукционного алгоритма от \( \varepsilon \) для различных радиусов видимости.}
    \label{fig:epsilon_accuracy_chart}
\end{figure}

Наблюдения:
\begin{itemize}
    \item Время выполнения уменьшается с ростом \( \varepsilon \), так как требуется меньше итераций.
    \item Точность падает при больших \( \varepsilon \), но при \( \varepsilon \leq 10^{-2} \) превышает 90\%.
\end{itemize}

\subsection{Точность по итерациям}

\vspace{0.3cm}

Точность на каждой итерации исследовалась для \( n \in \{10, 30, 50, 100\} \) и \( r_v = 200 \) (см. рисунок \ref{fig:acc_per_iteration_chart}).

\begin{figure}[h]
    \centering
    \includegraphics[width=0.8\textwidth]{my_folder/images/accuracy_per_iteration_chart.png}
    \caption{Зависимость относительной точности аукционного алгоритма от номера итерации для различных размеров задачи.}
    \label{fig:acc_per_iteration_chart}
\end{figure}

Наблюдения:
\begin{itemize}
    \item Точность быстро растет на первых итерациях, стабилизируясь на уровне 90--100\%.
    \item Для больших \( n \) требуется больше итераций из-за увеличения числа конфликтов.
\end{itemize}

\section{Анализ результатов}

Результаты подтверждают гипотезы, сформулированные в главе \ref{ch2}:

\begin{itemize}
    \item \textbf{Гипотеза 1}: Алгоритм завершается за конечное число шагов, а время выполнения пропорционально \( C / \varepsilon \), что подтверждается зависимостью времени от \( \varepsilon \) и \( r_v \).
    \item \textbf{Гипотеза 2}: Локальная оптимальность достигается в каждой связной компоненте при \( \varepsilon < 1/n \), но глобальная оптимальность ограничена числом компонент.
    \item \textbf{Гипотеза 3}: Высокая точность при небольших \( r_v \) указывает на потенциальную устойчивость в динамических сетях в условиях ограниченной связи.
\end{itemize}

Модифицированный аукционный алгоритм демонстрирует конкурентоспособное время выполнения и высокую точность (95--100\%) при подходящих параметрах.



\section{Выводы}

Модифицированный аукционный алгоритм показал эффективность в распределенных системах роботов, обеспечивая высокую точность и конкурентоспособное время выполнения. Параллельная обработка связных компонент снижает вычислительные затраты. Дальнейшие исследования могут быть направлены на тестирование алгоритма в динамических сетях и улучшение глобальной оптимальности.