\chapter{Разработка модифицированного аукционного алгоритма}
\label{ch2}

В данной главе представлена разработка модифицированного аукционного алгоритма для распределенных систем роботов с ограниченными областями связи. На основе анализа аукционного алгоритма Бертсекаса \cite{bertsekas1990} предлагается модификация, учитывающая топологические ограничения сети, при которых роботы обмениваются информацией только с соседями, объединенными в группы на основе радиуса связи. Каждый робот имеет доступ ко всем целям, что упрощает задачу назначения, но требует координации внутри групп. Описывается математическая модель задачи с учетом локальной доступности данных, а также алгоритмические изменения, обеспечивающие независимую обработку связных компонент и устойчивость к отсутствию коммуникации между группами. Формулируются гипотезы об эффективности и близости решения к оптимальному.

\section{Математическая модель задачи}

Задача назначения формулируется следующим образом. Дано множество из \( n \) роботов \( R = \{r_1, r_2, \ldots, r_n\} \) и множество из \( m \) целей \( T = \{t_1, t_2, \ldots, t_m\} \). Каждый робот \( r_i \) имеет координаты \( p_i = (x_i, y_i) \), а каждая цель \( t_j \) --- координаты \( q_j = (x_j, y_j) \). Выгода от назначения робота \( r_i \) цели \( t_j \) задана матрицей \( \{\alpha_{ij}\} \), где \( \alpha_{ij} \) зависит от расстояния между \( p_i \) и \( q_j \), например, обратно пропорциональна евклидову расстоянию:
\[
d(p_i, q_j) = \sqrt{(x_j - x_i)^2 + (y_j - y_i)^2}.
\]
Все роботы имеют доступ ко всем целям, поэтому \( \alpha_{ij} \) определена для любых \( i \) и \( j \).

Матрица видимости роботов \( V = \{v_{ij}\} \) определяет топологию сети: \( v_{ij} = 1 \), если \( d(p_i, p_j) \leq r_v \) (роботы \( r_i \) и \( r_j \) могут обмениваться информацией), и \( v_{ij} = 0 \) в противном случае. Роботы объединяются в группы (связные компоненты графа \( V \)), внутри которых проводится аукцион.

Цель задачи --- максимизировать суммарную выгоду:
\[
\sum_{i=1}^n \alpha_{i j_i} \to \max,
\]
где \( j_i \) --- цель, назначенная роботу \( r_i \), при условиях:
\begin{itemize}
    \item Цель может быть назначена нескольким роботам в процессе аукциона, но в итоговом решении для каждой цели \( t_j \) учитывается только робот с максимальной выгодой \( \alpha_{ij} \), определяемой как наибольшая эффективность (например, минимальное время достижения цели). Остальные назначения для этой цели отбрасываются.
    \item Роботы обмениваются информацией о ценах и назначениях только внутри связных компонент графа видимости \( V \).
\end{itemize}

\section{Модификация аукционного алгоритма}

Модифицированный аукционный алгоритм адаптирует подход Бертсекаса \cite{bertsekas1990} для учета отсутствия коммуникации между роботами, за исключением обмена внутри связных компонент. Ключевые изменения включают:
\begin{itemize}
    \item Полная видимость целей: Все цели доступны каждому роботу, поэтому \( \alpha_{ij} \) определена для всех \( i, j \).
    \item Разделение на связные компоненты: Роботы группируются по графу видимости \( V \) с использованием поиска в ширину, что позволяет проводить аукцион независимо для каждой компоненты.
    \item Независимая обработка компонент: Каждая связная компонента обрабатывается отдельно, моделируя распределенную систему без коммуникации между группами.
    \item Разрешение конфликтов: Если несколько роботов из разных компонент выбирают одну цель, учитывается только робот с максимальной выгодой \( \alpha_{ij} \), а остальные назначения отбрасываются.
\end{itemize}

\subsection{Описание алгоритма}

Алгоритм состоит из следующих шагов:
\begin{enumerate}
    \item \textbf{Инициализация}:
    \begin{itemize}
        \item Задать матрицу выгод \( \{\alpha_{ij}\} \), координаты роботов \( \{p_i\} \) и целей \( \{q_j\} \), радиус видимости \( r_v \).
        \item Построить матрицу видимости роботов \( V \): \( v_{ij} = 1 \), если \( d(p_i, p_j) \leq r_v \), и \( 0 \) иначе.
        \item Инициализировать цены целей \( \{p_j = 0\} \) и назначения \( \{j_i = -1\} \) (роботы изначально не назначены).
    \end{itemize}
    \item \textbf{Разделение на компоненты}:
    \begin{itemize}
        \item Используя поиск в ширину на графе \( V \), найти связные компоненты \( C_1, C_2, \ldots, C_k \), где роботы в одной компоненте могут обмениваться данными.
    \end{itemize}
    \item \textbf{Аукцион в компонентах}:
    \begin{itemize}
        \item Для каждой компоненты \( C_l \):
        \begin{enumerate}
            \item Для каждого робота \( r_i \in C_l \) вычислить текущую прибыль: \( \alpha_{i j_i} - p_{j_i} \), если \( j_i \neq -1 \), иначе \( 0 \).
            \item Проверить условие почти счастья: \( \alpha_{i j_i} - p_{j_i} \geq \max_j \{\alpha_{ij} - p_j\} - \varepsilon \), где \( \varepsilon > 0 \).
            \item Если робот не почти счастлив, найти цель \( t_{j_i} \): \( j_i = \arg \max_j \{\alpha_{ij} - p_j\} \), включая фиктивную цель (\( \alpha_{i,-1} = 0 \)).
            \item Вычислить \( v_i = \max_j \{\alpha_{ij} - p_j\} \), \( w_i = \max_{j \neq j_i} \{\alpha_{ij} - p_j\} \).
            \item Если \( j_i = -1 \), переназначить робота на фиктивную цель без изменения цен.
            \item Иначе переназначить: робот \( r_i \) получает цель \( t_{j_i} \), прежний владелец \( t_{j_i} \) (если есть) становится неназначенным.
            \item Увеличить цену: \( p_{j_i} += v_i - w_i + \varepsilon \).
            \item Повторять, пока все роботы в \( C_l \) не станут почти счастливы.
        \end{enumerate}
    \end{itemize}
    \item \textbf{Разрешение конфликтов}:
    \begin{itemize}
        \item Для каждой цели \( t_j \) собрать всех роботов \( \{r_i\} \), назначенных на \( t_j \) из разных компонент.
        \item Выбрать робота \( r_i \) с максимальной выгодой \( \alpha_{ij} \), определяемой как наибольшая эффективность (например, минимальное время достижения цели). Остальные роботы, назначенные на \( t_j \), переназначаются на фиктивную цель (\( j_i = -1 \)).
    \end{itemize}
    \item \textbf{Вычисление результата}:
    \begin{itemize}
        \item Вычислить суммарную выгоду: \( \sum_{i=1}^n \alpha_{i j_i} \), где \( j_i \neq -1 \), учитывая только назначения, выбранные после разрешения конфликтов.
        \item Вернуть назначения \( \{j_i\} \) и суммарную выгоду.
    \end{itemize}
\end{enumerate}

\subsection{Сходимость алгоритма}

\begin{theorem}[Сходимость модифицированного аукционного алгоритма]
\label{thm:mod_auction_convergence}
Модифицированный аукционный алгоритм завершается за конечное число шагов, при котором все роботы в каждой связной компоненте удовлетворяют условию почти счастья, определенному в главе \ref{ch:analysis}.
\end{theorem}

\textbf{Доказательство}. 
Модифицированный алгоритм проводит аукцион в каждой связной компоненте \( C_l \), следуя шагам оригинального аукционного алгоритма, описанного в главе \ref{ch:analysis} \cite{bertsekas1990}. Согласно теореме \ref{thm:auction_convergence} (глава \ref{ch:analysis}), оригинальный аукционный алгоритм завершается за конечное число шагов, обеспечивая почти равновесие, при котором все назначенные роботы почти счастливы, то есть для каждого робота \( i \), назначенного цели \( j_i \), выполняется:
\[
\alpha_{i j_i} - p_{j_i} \geq \max_{j=1,\ldots,m} \{\alpha_{ij} - p_j\} - \varepsilon.
\]
Поскольку аукцион в каждой компоненте \( C_l \) идентичен оригинальному, сходимость внутри \( C_l \) гарантируется той же теоремой. Разрешение конфликтов после аукциона не влияет на сходимость, так как оно происходит постфактум и лишь переназначает роботов на фиктивную цель (\( j_i = -1 \)) без изменения цен. Таким образом, алгоритм завершается за конечное число шагов, обеспечивая почти равновесие в каждой компоненте.

\subsection{Оценка итераций}

\begin{claim}[Оценка итераций]
\label{claim:mod_auction_iterations}
Общее число итераций модифицированного аукционного алгоритма ограничено \( O\left( \frac{C}{\varepsilon} \right) \), где \( C = \max_{i,j} |\alpha_{ij}| \).
\end{claim}

\textbf{Доказательство}. 
В каждой связной компоненте \( C_l \) модифицированный алгоритм выполняет аукцион, идентичный оригинальному, описанному в главе \ref{ch:analysis} \cite{bertsekas1990}. Согласно утверждению \ref{claim:auction_iterations} (глава \ref{ch:analysis}), число итераций оригинального аукционного алгоритма пропорционально \( C / \varepsilon \), где \( C = \max_{i,j} |\alpha_{ij}| \). Поскольку аукцион в каждой компоненте \( C_l \) следует той же логике, число итераций в \( C_l \) также составляет \( O(C / \varepsilon) \). Общее число итераций определяется максимальным числом итераций среди всех компонент, то есть \( O(C / \varepsilon) \).

\subsection{Оптимальность}

\begin{theorem}[Оптимальность модифицированного алгоритма]
\label{thm:mod_auction_optimality}
В каждой связной компоненте \( C_l \) модифицированный аукционный алгоритм при целочисленных выгодах \( \alpha_{ij} \) и \( \varepsilon < 1/n \) дает оптимальное решение локальной задачи аукциона, как определено в главе \ref{ch:analysis}. Однако общее решение после объединения компонент не гарантирует глобальной оптимальности из-за отсутствия коммуникации между компонентами.
\end{theorem}

\textbf{Доказательство}. 
В каждой связной компоненте \( C_l \) аукцион проводится по правилам оригинального аукционного алгоритма, описанного в главе \ref{ch:analysis} \cite{bertsekas1990}. Согласно теореме \ref{thm:auction_optimality} (глава \ref{ch:analysis}), при целочисленных \( \alpha_{ij} \) и \( \varepsilon < 1/n \) оригинальный аукционный алгоритм обеспечивает оптимальное решение, где суммарная выгода \( \sum_{i \in C_l} \alpha_{i j_i} \) равна локальному оптимуму. Таким образом, в каждой компоненте \( C_l \) достигается оптимальное локальное назначение. 

Однако глобальная оптимальность не гарантируется. После аукциона в компонентах разрешение конфликтов выбирает для каждой цели \( t_j \) робота с максимальной выгодой \( \alpha_{ij} \) среди назначенных, переназначая остальных на фиктивную цель. Из-за отсутствия коммуникации между компонентами роботы в \( C_l \) не учитывают выгоды роботов из других компонент, что может привести к выбору локально оптимальных, но глобально неоптимальных назначений. Следовательно, суммарная выгода \( \sum_{i=1}^n \alpha_{i j_i} \) может быть меньше глобального оптимума.

\section{Гипотезы об эффективности}

Для оценки модифицированного аукционного алгоритма сформулированы следующие гипотезы:
\begin{itemize}
    \item \textbf{Гипотеза 1}: Алгоритм завершается за конечное число шагов, и время выполнения пропорционально \( C / \varepsilon \), где \( C = \max_{i,j} |\alpha_{ij}| \), с уменьшением коммуникационных затрат за счет независимой обработки связных компонент.
    \item \textbf{Гипотеза 2}: В каждой связной компоненте при целочисленных выгодах \( \alpha_{ij} \) и \( \varepsilon < 1/n \) достигается оптимальное решение, но отклонение от глобального оптимума зависит от числа связных компонент \( k \) и радиуса видимости \( r_v \).
    \item \textbf{Гипотеза 3}: Алгоритм эффективен в динамических сетях с изменяющимися позициями роботов при периодическом обновлении матрицы видимости \( V \).
\end{itemize}

\section{Выводы}

Разработан модифицированный аукционный алгоритм, учитывающий отсутствие коммуникации между роботами в распределенных системах, за исключением обмена внутри связных компонент. Алгоритм группирует роботов в связные компоненты с помощью матрицы видимости \( V \), проводит аукцион независимо в каждой компоненте и обеспечивает полную видимость целей. Доказана сходимость за конечное число шагов (теорема \ref{thm:mod_auction_convergence}) и оптимальность в каждой компоненте для целочисленных выгод \( \alpha_{ij} \) при \( \varepsilon < 1/n \) (теорема \ref{thm:mod_auction_optimality}), но глобальная оптимальность не достигается из-за отсутствия коммуникации между компонентами. Преимущества включают простоту реализации, снижение коммуникационных затрат и адаптивность к реальным сценариям робототехники. Сформулированные гипотезы задают направления для дальнейших исследований эффективности и устойчивости алгоритма.