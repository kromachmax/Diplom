\chapter{Разработка модифицированного аукционного алгоритма}
\label{ch2}

В данной главе представлена разработка модифицированного аукционного алгоритма для распределенных систем роботов с ограниченными областями связи. На основе анализа аукционного алгоритма Бертсекаса \cite{bertsekas1990} предлагается модификация, учитывающая топологические ограничения сети, при которых роботы обмениваются информацией только с соседями, объединенными в группы на основе радиуса связи. Каждый робот имеет доступ ко всем задачам, что упрощает задачу назначения, но требует координации внутри групп. Описывается математическая модель задачи с учетом локальной доступности данных, а также алгоритмические изменения, обеспечивающие параллельную обработку и устойчивость к ограниченной коммуникации. Формулируются гипотезы об эффективности и близости решения к оптимальному.

\section{Математическая модель задачи}

Задача назначения формулируется следующим образом. Дано множество из \( n \) роботов \( R = \{r_1, r_2, \ldots, r_n\} \) и множество из \( m \) задач \( T = \{t_1, t_2, \ldots, t_m\} \). Каждый робот \( r_i \) имеет координаты \( p_i = (x_i, y_i) \), а каждая задача \( t_j \) --- координаты \( q_j = (x_j, y_j) \). Выгода от назначения робота \( r_i \) задаче \( t_j \) задана матрицей \( \{\alpha_{ij}\} \), где \( \alpha_{ij} \) зависит от расстояния между \( p_i \) и \( q_j \), например, обратно пропорциональна евклидову расстоянию:
\[
d(p_i, q_j) = \sqrt{(x_j - x_i)^2 + (y_j - y_i)^2}.
\]
Все роботы имеют доступ ко всем задачам, поэтому \( \alpha_{ij} \) определена для любых \( i \) и \( j \).

Матрица видимости роботов \( V = \{v_{ij}\} \) определяет топологию сети: \( v_{ij} = 1 \), если \( d(p_i, p_j) \leq r_v \) (роботы \( r_i \) и \( r_j \) могут обмениваться информацией), и \( v_{ij} = 0 \) в противном случае. Роботы объединяются в группы (связные компоненты графа \( V \)), внутри которых проводится аукцион.

Цель задачи --- максимизировать суммарную выгоду:
\[
\sum_{i=1}^n \alpha_{i j_i} \to \max,
\]
где \( j_i \) --- задача, назначенная роботу \( r_i \), при условиях:
\begin{itemize}
    \item Задача может быть назначена нескольким роботам в процессе аукциона, но в итоговом решении для каждой задачи \( t_j \) учитывается только робот с максимальной выгодой \( \alpha_{ij} \), определяемой как наибольшая эффективность (например, минимальное время достижения цели). Остальные назначения для этой задачи отбрасываются.
    \item Роботы обмениваются информацией о ценах и назначениях только внутри связных компонент графа видимости \( V \).
\end{itemize}

\section{Модификация аукционного алгоритма}

Модифицированный аукционный алгоритм адаптирует подход Бертсекаса \cite{bertsekas1990} для учета ограниченной коммуникации между роботами. Ключевые изменения включают:
\begin{itemize}
    \item Полная видимость задач: Все задачи доступны каждому роботу, поэтому \( \alpha_{ij} \) определена для всех \( i, j \).
    \item Разделение на связные компоненты: Роботы группируются по графу видимости \( V \) с использованием поиска в ширину, что позволяет проводить аукцион независимо для каждой компоненты.
    \item Параллельная обработка: Каждая связная компонента обрабатывается одновременно, моделируя распределенную систему.
    \item Разрешение конфликтов: Если несколько роботов из разных компонент выбирают одну задачу, учитывается только робот с максимальной выгодой \( \alpha_{ij} \), а остальные назначения отбрасываются.
\end{itemize}

\subsection{Описание алгоритма}

Алгоритм состоит из следующих шагов:
\begin{enumerate}
    \item \textbf{Инициализация}:
    \begin{itemize}
        \item Задать матрицу выгод \( \{\alpha_{ij}\} \), координаты роботов \( \{p_i\} \) и задач \( \{q_j\} \), радиус видимости \( r_v \).
        \item Построить матрицу видимости роботов \( V \): \( v_{ij} = 1 \), если \( d(p_i, p_j) \leq r_v \), и \( 0 \) иначе.
        \item Инициализировать цены задач \( \{p_j = 0\} \) и назначения \( \{j_i = -1\} \) (роботы изначально не назначены).
    \end{itemize}
    \item \textbf{Разделение на компоненты}:
    \begin{itemize}
        \item Используя поиск в ширину на графе \( V \), найти связные компоненты \( C_1, C_2, \ldots, C_k \), где роботы в одной компоненте могут обмениваться данными.
    \end{itemize}
    \item \textbf{Параллельный аукцион}:
    \begin{itemize}
        \item Для каждой компоненты \( C_l \):
        \begin{enumerate}
            \item Для каждого робота \( r_i \in C_l \) вычислить текущую прибыль: \( \alpha_{i j_i} - p_{j_i} \), если \( j_i \neq -1 \), иначе \( 0 \).
            \item Проверить условие почти счастья: \( \alpha_{i j_i} - p_{j_i} \geq \max_j \{\alpha_{ij} - p_j\} - \varepsilon \), где \( \varepsilon > 0 \).
            \item Если робот не почти счастлив, найти задачу \( t_{j_i} \): \( j_i = \arg \max_j \{\alpha_{ij} - p_j\} \), включая фиктивную задачу (\( \alpha_{i,-1} = 0 \)).
            \item Вычислить \( v_i = \max_j \{\alpha_{ij} - p_j\} \), \( w_i = \max_{j \neq j_i} \{\alpha_{ij} - p_j\} \).
            \item Если \( j_i = -1 \), переназначить робота на фиктивную задачу без изменения цен.
            \item Иначе переназначить: робот \( r_i \) получает задачу \( t_{j_i} \), прежний владелец \( t_{j_i} \) (если есть) становится неназначенным.
            \item Увеличить цену: \( p_{j_i} += v_i - w_i + \varepsilon \).
            \item Повторять, пока все роботы в \( C_l \) не станут почти счастливы.
        \end{enumerate}
    \end{itemize}
    \item \textbf{Разрешение конфликтов}:
    \begin{itemize}
        \item Для каждой задачи \( t_j \) собрать всех роботов \( \{r_i\} \), назначенных на \( t_j \) из разных компонент.
        \item Выбрать робота \( r_i \) с максимальной выгодой \( \alpha_{ij} \), определяемой как наибольшая эффективность (например, минимальное время достижения цели). Остальные роботы, назначенные на \( t_j \), переназначаются на фиктивную задачу (\( j_i = -1 \)).
    \end{itemize}
    \item \textbf{Вычисление результата}:
    \begin{itemize}
        \item Вычислить суммарную выгоду: \( \sum_{i=1}^n \alpha_{i j_i} \), где \( j_i \neq -1 \), учитывая только назначения, выбранные после разрешения конфликтов.
        \item Вернуть назначения \( \{j_i\} \) и суммарную выгоду.
    \end{itemize}
\end{enumerate}

% Переработанные доказательства для раздела "Модификация аукционного алгоритма" в chapter2.tex

\subsection{Сходимость алгоритма}

\begin{theorem}[Сходимость модифицированного аукционного алгоритма]
\label{thm:mod_auction_convergence}
Модифицированный аукционный алгоритм завершается за конечное число шагов, при котором все роботы в каждой связной компоненте удовлетворяют условию почти счастья, определенному в главе \ref{ch:analysis}.
\end{theorem}

\textbf{Доказательство}. 
Модифицированный алгоритм разделяет роботов на связные компоненты \( C_1, C_2, \ldots, C_k \) на основе матрицы видимости \( V \). В каждой компоненте \( C_l \) аукцион проводится независимо, следуя структуре оригинального аукционного алгоритма, описанного в главе \ref{ch:analysis} \cite{bertsekas1990}. Согласно теореме \ref{thm:auction_convergence} (глава \ref{ch:analysis}), оригинальный аукционный алгоритм сходится за конечное число шагов, обеспечивая почти равновесие, при котором все агенты почти счастливы, то есть для каждого агента \( i \), назначенного объекту \( j_i \), выполняется:
\[
a_{i j_i} - p_{j_i} \geq \max_{j=1,\ldots,n} \{a_{ij} - p_j\} - \varepsilon,
\]
где \( \varepsilon > 0 \) — параметр точности, \( a_{ij} \) — выгода, \( p_j \) — цена объекта.

В модифицированном алгоритме для каждого робота \( r_i \in C_l \), назначенного задаче \( t_{j_i} \), обновление цен происходит аналогично: после ставки цена \( p_{j_i} \) увеличивается на \( \gamma_i = v_i - w_i + \varepsilon \), где \( v_i = \max_j \{\alpha_{ij} - p_j\} \), \( w_i = \max_{j \neq j_i} \{\alpha_{ij} - p_j\} \). Как показано в доказательстве теоремы \ref{thm:auction_convergence}, это обновление обеспечивает, что:
\[
\alpha_{i j_i} - (p_{j_i} + \gamma_i) = w_i - \varepsilon \leq \max_{k \neq j_i} \{\alpha_{i k} - p_k\} - \varepsilon,
\]
удовлетворяя условию почти счастья. Поскольку цены \( p_k \) для \( k \neq j_i \) не уменьшаются, робот \( r_i \) остается почти счастливым, пока удерживает задачу \( t_{j_i} \).

Конечность числа шагов в каждой компоненте следует из теоремы \ref{thm:auction_convergence}: число роботов и задач конечно, а цены увеличиваются минимум на \( \varepsilon \). Если задача \( t_j \) получает \( m \) ставок, то \( p_j \geq m \varepsilon \). При большом \( m \) \( \alpha_{i j} - p_j \) становится меньше \( \alpha_{i k} - p_k \) для задачи \( k \) без ставок, побуждая роботов выбрать другие задачи. Параллельная обработка компонент не влияет на сходимость, так как каждая компонента независима. Разрешение конфликтов, при котором для задачи \( t_j \) выбирается робот с максимальной выгодой \( \alpha_{ij} \), происходит после завершения аукциона в компонентах и не нарушает почти счастье внутри них, так как переназначение на фиктивную задачу (\( j_i = -1 \)) не требует изменения цен. Таким образом, алгоритм завершается за конечное число шагов, обеспечивая почти равновесие в каждой компоненте.

\subsection{Оценка итераций}

\begin{claim}[Оценка итераций]
\label{claim:mod_auction_iterations}
Общее число итераций модифицированного аукционного алгоритма ограничено \( O\left( \frac{C}{\varepsilon} \right) \), где \( C = \max_{i,j} |\alpha_{ij}| \).
\end{claim}

\textbf{Доказательство}. 
Модифицированный алгоритм выполняет аукцион в каждой связной компоненте \( C_l \), аналогичный оригинальному аукционному алгоритму, описанному в главе \ref{ch:analysis} \cite{bertsekas1990}. Согласно утверждению \ref{claim:auction_iterations} (глава \ref{ch:analysis}), число итераций в оригинальном алгоритме пропорционально \( C / \varepsilon \), где \( C = \max_{i,j} |a_{ij}| \). В компоненте \( C_l \) с \( |C_l| \) роботами и \( m \) задачами аукцион следует той же логике: каждая ставка увеличивает цену задачи минимум на \( \varepsilon \), и рост цен ограничен \( C = \max_{i,j} |\alpha_{ij}| \). Таким образом, число итераций в \( C_l \) составляет \( O(C / \varepsilon) \). Поскольку компоненты обрабатываются параллельно, общее число итераций определяется максимумом по всем компонентам, то есть \( O(C / \varepsilon) \).

\subsection{Оптимальность}

\begin{theorem}[Оптимальность модифицированного алгоритма]
\label{thm:mod_auction_optimality}
В каждой связной компоненте \( C_l \) модифицированный аукционный алгоритм при целочисленных выгодах \( \alpha_{ij} \) и \( \varepsilon < 1/n \) даёт оптимальное решение задачи аукциона, как определено в главе \ref{ch:analysis}. Однако общее решение после объединения компонент не гарантирует глобальной оптимальности из-за ограниченной коммуникации.
\end{theorem}

\textbf{Доказательство}. 
В каждой связной компоненте \( C_l \) модифицированный алгоритм выполняет аукцион, аналогичный оригинальному, описанному в главе \ref{ch:analysis} \cite{bertsekas1990}. Для робота \( r_i \in C_l \), назначенного задаче \( t_{j_i} \), достигается почти равновесие, где:
\[
\alpha_{i j_i} - p_{j_i} \geq \max_j \{\alpha_{ij} - p_j\} - \varepsilon.
\]
Согласно теореме \ref{thm:auction_optimality} (глава \ref{ch:analysis}), при целочисленных \( a_{ij} \) и \( \varepsilon < 1/n \) оригинальный аукционный алгоритм даёт оптимальное решение, так как суммарная выгода удовлетворяет:
\[
\sum_{i=1}^n a_{i j_i} = A^*,
\]
где \( A^* \) — оптимальная выгода. Аналогично, в компоненте \( C_l \) с \( |C_l| \) роботами и целочисленными \( \alpha_{ij} \), при \( \varepsilon < 1/n \) (где \( n \) — общее число роботов) суммарная выгода \( \sum_{i \in C_l} \alpha_{i j_i} \) оптимальна для локальной задачи аукциона.

Однако глобальная оптимальность не достигается. После аукциона в компонентах выполняется разрешение конфликтов, где для задачи \( t_j \), назначенной нескольким роботам из разных компонент, выбирается робот с максимальной выгодой \( \alpha_{ij} \), а остальные переназначаются на фиктивную задачу (\( \alpha_{i,-1} = 0 \)). Из-за ограниченной коммуникации между компонентами роботы в \( C_l \) не имеют доступа к данным о выгодах роботов из других компонент, что может привести к выбору локально оптимальных, но глобально неоптимальных назначений. Таким образом, суммарная выгода \( \sum_{i=1}^n \alpha_{i j_i} \) после разрешения конфликтов может быть меньше глобального оптимума \( A^* \).

\section{Гипотезы об эффективности}

Для оценки модифицированного аукционного алгоритма сформулированы следующие гипотезы:
\begin{itemize}
    \item \textbf{Гипотеза 1}: Алгоритм завершается за конечное число шагов, и время выполнения пропорционально \( C / \varepsilon \), где \( C = \max_{i,j} |\alpha_{ij}| \), с уменьшением времени за счёт параллельной обработки связных компонент.
    \item \textbf{Гипотеза 2}: В каждой связной компоненте при целочисленных выгодах \( \alpha_{ij} \) и \( \varepsilon < 1/n \) достигается оптимальное решение, но отклонение от глобального оптимума зависит от числа связных компонент \( k \) и радиуса видимости \( r_v \).
    \item \textbf{Гипотеза 3}: Алгоритм эффективен в динамических сетях с изменяющимися позициями роботов при периодическом обновлении матрицы видимости \( V \).
\end{itemize}

\section{Выводы}

Разработан модифицированный аукционный алгоритм, учитывающий ограниченные области связи в распределённых системах роботов. Алгоритм группирует роботов в связные компоненты с помощью матрицы видимости \( V \), проводит аукцион параллельно и обеспечивает полную видимость задач. Доказана сходимость за конечное число шагов (теорема \ref{thm:mod_auction_convergence}) и оптимальность в каждой компоненте для целочисленных выгод \( \alpha_{ij} \) при \( \varepsilon < 1/n \) (теорема \ref{thm:mod_auction_optimality}), но глобальная оптимальность не достигается из-за ограниченной коммуникации. Преимущества включают простоту реализации, снижение коммуникационных затрат и адаптивность к реальным сценариям робототехники. Сформулированные гипотезы задают направления для дальнейших исследований эффективности и устойчивости алгоритма.