\chapter*{Введение}
\addcontentsline{toc}{chapter}{Введение}

Развитие робототехники и распределенных систем управления в последние десятилетия привело к значительному росту интереса к задачам оптимального распределения ресурсов в условиях ограниченной коммуникации. Такие задачи, известные как задачи назначения, требуют эффективного распределения ограниченного набора ресурсов (например, задач или объектов) между агентами (роботами) для максимизации общей выгоды. В условиях ограниченных областей связи, когда не все роботы могут взаимодействовать друг с другом или отсутствует единый центр управления, традиционные алгоритмы, такие как венгерский метод, сталкиваются с вычислительными трудностями или вообще неприменимы. Это подчеркивает необходимость разработки новых алгоритмов, способных учитывать топологические ограничения и обеспечивать высокую эффективность.

\textbf{Актуальность исследования} обусловлена растущей потребностью в автоматизации сложных систем управления в робототехнике, где ограниченные области связи создают дополнительные проблемы для координации роботов. По данным исследований \cite{bertsekas1990}, аукционные алгоритмы демонстрируют высокую эффективность для задач назначения в распределенных системах, однако их применение в робототехнике с учетом топологических ограничений остается недостаточно изученным. Это определяет актуальность данной работы, направленной на разработку и исследование новых подходов к решению задач назначения.

\textbf{Объект исследования} --- распределенные системы управления роботами, функционирующие в условиях ограниченной коммуникации.

\textbf{Предмет исследования} --- алгоритмы назначения, обеспечивающие оптимальное распределение целей между роботами с учетом ограниченных областей связи.

\textbf{Цель исследования} --- повышение эффективности распределения целей в распределенных системах роботов с ограниченными областями связи путем разработки и применения модифицированного аукционного алгоритма.

\textbf{Задачи исследования}:
\begin{enumerate}
    \item Изучить существующие алгоритмы решения задачи назначения, включая аукционный и венгерский методы.
    \item Проанализировать влияние ограниченных областей связи на эффективность алгоритмов назначения.
    \item Разработать модификацию аукционного алгоритма, учитывающую топологические ограничения в системах роботов.
    \item Реализовать предложенный алгоритм и венгерский алгоритм в программной среде.
    \item Провести сравнительное тестирование алгоритмов на модельных задачах с различными характеристиками коммуникационных ограничений.
\end{enumerate}

\textbf{Теоретическая база исследования} включает работы по задачам назначения и групповому управлению роботами. Основой послужили исследования Д. Бертсекаса \cite{bertsekas1990}, описывающие аукционный алгоритм и его преимущества, а также работы Х. Куна \cite{kuhn1955} по венгерскому методу. Значительное внимание уделено работам по распределённым системам управления \cite{pshikhopov2015},  \cite{kalyaev2009}, а также проблемам оптимального распределения целей в мультироботных системах \cite{gerkey2003}. В процессе подготовки исследования были изучены такие дисциплины, как «Алгоритмы и структуры данных», «Робототехника» и «Теория оптимизации».

\textbf{Методологическая база} включает общенаучные методы (анализ, моделирование, эксперимент) и конкретно-научные методы (методы линейного программирования, венгерский алгоритм, аукционные алгоритмы, алгоритмы на графах). В работе применены подходы к распараллеливанию
 вычислений, а также методы сравнительного анализа алгоритмов.

\textbf{Информационная база} включает материалы учебных дисциплин, данные из научных публикаций, а также результаты моделирования, полученные в ходе выполнения данной ВКР.

\textbf{Степень научной разработанности} проблемы характеризуется значительным вниманием к задачам назначения в работах Д. Бертсекаса, Х. Куна и других авторов \cite{bertsekas1990}-\cite{gerkey2003}. Исследования группового управления роботами, включая работы В.Х. Пшихопова \cite{pshikhopov2015} и И.А. Каляева \cite{kalyaev2009}, подчеркивают важность учета коммуникационных ограничений. Однако адаптация аукционных алгоритмов к ограниченным областям связи в робототехнике остается малоисследованной, что определяет необходимость разработки новых подходов.

\textbf{Научная новизна} заключается в разработке модифицированного аукционного алгоритма, адаптированного для распределенных систем роботов с ограниченными областями связи. Новизна проявляется в учете топологических ограничений коммуникации, что позволяет повысить эффективность алгоритма по сравнению с традиционными методами. Также новизна состоит в сравнительном анализе аукционного и венгерского алгоритмов в контексте робототехнических приложений.

\textbf{Практическая значимость} заключается в возможности применения разработанного алгоритма для оптимизации распределения задач в системах управления роботами, работающих в условиях ограниченной коммуникации, таких как складская логистика, автономные транспортные системы и роевые робототехнические платформы. Результаты работы могут быть использованы для повышения эффективности реальных систем и дальнейших исследований в области распределенного управления.

\textbf{Апробация результатов} включает представление результатов на защите ВКР и размещение работы на портале СПбПУ.

Введение определяет рамки исследования, обосновывая актуальность проблемы оптимального распределения целей в распределённых системах управления роботами, особенно в условиях ограниченной коммуникации. Оно формулирует цель работы — повышение эффективности распределения задач путём разработки нового подхода, основанного на аукционном алгоритме, а также связывает поставленные задачи с главами работы, задавая структуру исследования.

Первая глава посвящена анализу существующих алгоритмов назначения. В ней рассматриваются классические методы, их математические основы, преимущества и ограничения, с акцентом на их применимость в робототехнических системах. Особое внимание уделяется анализу их применимости в условиях, где связи между роботами ограничены, что подчёркивает необходимость нового подхода.

Вторая глава сосредоточена на разработке модифицированного аукционного алгоритма. Она описывает математическую модель задачи, учитывающую топологические ограничения сети, и представляет изменения, внесённые в классический аукционный метод, чтобы адаптировать его к условиям ограниченной связи. Также обсуждаются теоретические аспекты, включая сходимость и оценку производительности алгоритма.

Третья глава посвящена программной реализации предложенного модифицированного алгоритма и его классического аналога. В ней описывается используемая программная среда, структура кода и подходы к моделированию распределённых систем. Особое внимание уделяется методам тестирования и визуализации результатов, которые позволяют оценить поведение алгоритмов при различных сценариях.

Четвёртая глава фокусируется на тестировании и сравнительном анализе разработанного алгоритма с классическим методом. Она представляет результаты вычислительных экспериментов, демонстрирующих влияние ключевых параметров на производительность и точность. Графики и численные данные иллюстрируют, как алгоритм справляется с задачами в условиях ограниченной коммуникации.

Пятая глава обсуждает перспективы дальнейшего развития исследования. В ней рассматриваются возможные улучшения алгоритма. Также анализируются потенциальные области применения, подчёркивая практическую значимость разработки.

Заключение подводит итоги исследования, суммируя достигнутые результаты. Оно также указывает на направления для будущих исследований, которые могут расширить применимость разработанного подхода.
