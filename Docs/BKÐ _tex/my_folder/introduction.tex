\chapter*{Введение}
\addcontentsline{toc}{chapter}{Введение}

Развитие робототехники и распределенных систем управления в последние десятилетия привело к значительному росту интереса к задачам оптимального распределения ресурсов в условиях ограниченной коммуникации. Такие задачи, известные как задачи назначения, требуют эффективного распределения ограниченного набора ресурсов (например, задач или объектов) между агентами (роботами) для максимизации общей выгоды. В условиях ограниченных областей связи, когда не все роботы могут взаимодействовать друг с другом или отсутствует единый центр управления, традиционные алгоритмы, такие как венгерский метод, сталкиваются с вычислительными трудностями. Это подчеркивает необходимость разработки новых алгоритмов, способных учитывать топологические ограничения и обеспечивать высокую эффективность.

\textbf{Актуальность исследования} обусловлена растущей потребностью в автоматизации сложных систем управления в робототехнике, где ограниченные области связи создают дополнительные проблемы для координации роботов. По данным исследований \cite{bertsekas1990}, аукционные алгоритмы демонстрируют высокую эффективность для задач назначения, однако их применение в робототехнике с учетом топологических ограничений остается недостаточно изученным. Это определяет актуальность данной работы, направленной на разработку и исследование новых подходов к решению задач назначения.

\textbf{Объект исследования} --- распределенные системы управления роботами, функционирующие в условиях ограниченной коммуникации.

\textbf{Предмет исследования} --- алгоритмы назначения, обеспечивающие оптимальное распределение задач между роботами с учетом ограниченных областей связи.

\textbf{Цель исследования} --- разработать модифицированный аукционный алгоритм для задачи назначения в распределенных системах роботов с ограниченными областями связи, исследовать его и сравнить его эффективность с венгерским алгоритмом.

\textbf{Задачи исследования}:
\begin{enumerate}
    \item Изучить существующие алгоритмы решения задачи назначения, включая аукционный и венгерский методы.
    \item Проанализировать влияние ограниченных областей связи на эффективность алгоритмов назначения.
    \item Разработать модификацию аукционного алгоритма, учитывающую топологические ограничения в системах роботов.
    \item Реализовать предложенный алгоритм и венгерский алгоритм в программной среде.
    \item Провести сравнительное тестирование алгоритмов на модельных задачах с различными характеристиками коммуникационных ограничений.
\end{enumerate}

\textbf{Теоретическая база исследования} включает работы по задачам назначения и групповому управлению роботами. Основой послужили исследования Д. Бертсекаса \cite{bertsekas1990}, описывающие аукционный алгоритм и его преимущества, а также работы Х. Куна \cite{kuhn1955} по венгерскому методу. Значительное внимание уделено работам по распределённым системам управления \cite{pshikhopov2015, kalyaev2009}, а также проблемам оптимального распределения задач в мультироботных системах \cite{gerkey2003}. В процессе подготовки исследования были изучены такие дисциплины, как «Алгоритмы и структуры данных», «Робототехника» и «Теория оптимизации».

\textbf{Методологическая база} включает общенаучные методы (анализ, моделирование, эксперимент) и конкретно-научные методы (методы линейного программирования, аукционные методы, методы графов). В работе применены подходы параллельных вычислений, описанные в \cite{bertsekas1990}, а также методы сравнительного анализа алгоритмов.

\textbf{Информационная база} включает материалы учебных дисциплин, данные из научных публикаций, а также результаты моделирования, полученные в ходе выполнения данной ВКР.

\textbf{Степень научной разработанности} проблемы характеризуется значительным вниманием к задачам назначения в работах Д. Бертсекаса, Х. Куна и других авторов \cite{bertsekas1990, kuhn1955, bertsekas1989, gerkey2003}. Исследования группового управления роботами, включая работы В.Х. Пшихопова \cite{pshikhopov2015} и И.А. Каляева \cite{kalyaev2009}, подчеркивают важность учета коммуникационных ограничений. Однако адаптация аукционных алгоритмов к ограниченным областям связи в робототехнике остается малоисследованной, что определяет необходимость разработки новых подходов.

\textbf{Научная новизна} заключается в разработке модифицированного аукционного алгоритма, адаптированного для распределенных систем роботов с ограниченными областями связи. Новизна проявляется в учете топологических ограничений коммуникации, что позволяет повысить эффективность алгоритма по сравнению с традиционными методами. Также новизна состоит в сравнительном анализе аукционного и венгерского алгоритмов в контексте робототехнических приложений.

\textbf{Практическая значимость} заключается в возможности применения разработанного алгоритма для оптимизации распределения задач в системах управления роботами, работающих в условиях ограниченной коммуникации, таких как складская логистика, автономные транспортные системы и роевые робототехнические платформы. Результаты работы могут быть использованы для повышения эффективности реальных систем и дальнейших исследований в области распределенного управления.

\textbf{Апробация результатов} включает представление результатов на защите ВКР и размещение работы на портале СПбПУ.

Введение определяет рамки исследования, связывая поставленные задачи с главами работы. Первая глава посвящена анализу существующих алгоритмов назначения, вторая — разработке модифицированного аукционного алгоритма, третья — программной реализации, четвертая — тестированию и сравнению с венгерским алгоритмом.
