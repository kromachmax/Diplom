%%% Титульник ВКР / Thesis title 
%%
%% добавить лист в pdf-навигацию 
%% add to pdf navigation menu
%%
\pdfbookmark[-1]{\pdfTitle}{tit}
%%
\thispagestyle{empty}%
\makeatletter
\newgeometry{top=2cm,bottom=2cm,left=3cm,right=1cm,headsep=0cm,footskip=0cm}
\savegeometry{NoFoot}%
\makeatother



{\centering%
	\Ministry\\
	\SPbPU\\
	{%\bfseries %2020 - указание на изменения, которые могут быть введены в 2020 году
		Физико-механический институт\\
		Высшая школа прикладной математики и вычислительной физики}
\par}%


\vspace{0pt plus1fill} %число перед fill = кратность относительно некоторого расстояния fill, кусками которого заполнены пустые места



\noindent
\begin{minipage}{\linewidth}
	\vspace{\mfloatsep} % интервал 
	\begin{tabularx}{\linewidth}{Xl}
	&Работа допущена к защите     \\
	&Руководитель ОП     \\			
	&\underline{\hspace*{0.1\textheight}} К.Н. Козлов   \\
	&<<\underline{\hspace*{0.05\textheight}}>> \underline{\hspace*{0.1\textheight}} 
	2025~г.  \\ 
	\end{tabularx}
	\vspace{\mfloatsep} % интервал 	
\end{minipage}


\vspace{0pt plus2fill} %


{\centering%
	
	\MakeUppercase{\bfseries{}\DocType} \\ 
	\MakeUppercase{\thesisDegree}%


%\intervalS% %ОБЯЗАТЕЛЬНО ДОБАВИТЬ ОТСТУП, ЕСЛИ ХВАТАЕТ МЕСТА
{\centering%
	\MakeUppercase{\bfseries{ДЕЦЕНТРАЛИЗОВАННОЕ РЕШЕНИЕ ЗАДАЧИ О НАЗНАЧЕНИЯХ ЦЕЛЕЙ ГРУППЕ РОБОТОВ}}}%

}\par%

%\intervalS% %ОБЯЗАТЕЛЬНО ДОБАВИТЬ ОТСТУП, ЕСЛИ ХВАТАЕТ МЕСТА
%по специальности % для специалистов
\noindent	по направлению подготовки: 01.03.02 Прикладная математика и информатика\\% для бакалавров и магистров 
%\noindent Направленность  % для специалистов
\noindent	Направленность (профиль):	01.03.02\_02 Системное программирование % для бакалавров и магистров
% Лучше по~профилю, но что делать, так составили Положение
\par%





\vspace{4mm plus2fill}%

\noindent
\begin{tabularx}{\linewidth}{lXl}
	Выполнил              &	   &             \\
	студент гр.~5030102/10201     &    &
	М.А. Кромачев     \\[\mfloatsep]

	Руководитель 		  &    &             \\
	Доцент,		  &    &             \\
	кандидат физико-математических наук 	  &    & 
	И.Е. Ануфриев \\[\mfloatsep]
	
	Консультант		  &  & 			 \\
	Кандидат технических наук, \\ старший научный сотрудник, \\
	директор НТЦ разработки \\ программного обеспечения и СППР, АО НПП\\
	«Авиационная и морская электроника» 	  &    & С.А. Васильковский\\[\mfloatsep]
	
	Консультант  &    &  \\   	
	по нормоконтролю	&   & 
	Л.А. Арефьева  % обязателен
\end{tabularx} %


%
\vspace{0pt plus4fill}% 


\begin{center}%
Санкт-Петербург --- 2025\\
\end{center}%
\restoregeometry
\newpage