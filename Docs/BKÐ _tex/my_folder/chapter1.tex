\chapter{Анализ существующих алгоритмов назначения}
\label{ch:analysis}

\section{Введение в предметную область}
Задача назначения --- одна из ключевых проблем комбинаторной оптимизации, широко применяемая в робототехнике, логистике и управлении ресурсами. Согласно \cite{bertsekas1990}, задача заключается в распределении $n$ агентов (например, роботов) по $n$ объектам (например, задачам) для максимизации суммарной выгоды, заданной матрицей $\{a_{ij}\}$, где $a_{ij}$ --- выгода от назначения агента $i$ объекту $j$. Математически цель задачи:

\[
\sum_{i=1}^n a_{i j_i} \to \max,
\]

где $j_i$ --- объект, назначенный агенту $i$, и все $j_i$ различны.

В робототехнике задача назначения актуальна для распределения задач между роботами в условиях ограниченной коммуникации, когда роботы обмениваются информацией только с соседями. Такие ограничения, обусловленные топологией сети, требуют алгоритмов, эффективных в распределенных системах. Примеры приложений: складская логистика, автономные транспортные системы, роевые платформы \cite{kalyaev2009, gerkey2003}.

Цель главы --- проанализировать аукционный и венгерский алгоритмы, включая их математические основы, сходимость и оптимальность, и оценить их применимость к робототехнике. Анализ обосновывает необходимость модифицированного аукционного алгоритма для ограниченной коммуникации.

\section{Аукционный алгоритм}
Аукционный алгоритм, предложенный Д. Бертсекасом \cite{bertsekas1990}, представляет собой итеративный метод для решения задачи назначения, моделирующий процесс аукциона, где агенты делают ставки на объекты, а цены корректируются для достижения оптимального распределения.

\subsection{Постановка задачи}
Задача назначения (Assignment Problem) заключается в нахождении оптимального соответствия между $n$ агентами и $n$ объектами с учетом матрицы выгод $\{a_{ij}\}$, где $a_{ij}$ — выгода агента $i$ при назначении ему объекта $j$. Цель — максимизировать суммарную выгоду:
\[
\max \sum_{i=1}^n a_{i j_i},
\]
где $j_i$ — объект, назначенный агенту $i$, при условии, что каждые агент и объект участвуют ровно в одном назначении (биективное соответствие). Эта задача эквивалентна задаче линейного программирования с ограничениями:
\[
\begin{cases}
\sum_{i=1}^n x_{ij} = 1, & \forall j = 1, \ldots, n, \\
\sum_{j=1}^n x_{ij} = 1, & \forall i = 1, \ldots, n, \\
x_{ij} \in \{0, 1\}, & \forall i, j,
\end{cases}
\]
где $x_{ij} = 1$, если агент $i$ назначен объекту $j$, и $x_{ij} = 0$ в противном случае. Аукционный алгоритм решает эту задачу приближенно с заданной точностью $\varepsilon > 0$, достигая почти оптимального решения через итеративную корректировку цен.

\subsection{Описание алгоритма}
Согласно \cite{bertsekas1990}, алгоритм работает с матрицей выгод $\{a_{ij}\}$ и начинается с произвольного распределения и цен $\{p_j\}$. Ключевые определения:
\begin{itemize}
    \item \textit{Почти счастье}: Агент $i$, назначенный объекту $j_i$, почти счастлив, если:
    \[
    a_{i j_i} - p_{j_i} \geq \max_{j=1,\ldots,n} \{a_{ij} - p_j\} - \varepsilon,
    \]
    где $\varepsilon > 0$ — параметр точности.
    \item \textit{Почти равновесие}: Распределение и цены, при которых все агенты почти счастливы.
\end{itemize}

Шаги алгоритма \cite{bertsekas1990}:
\begin{enumerate}
    \item Проверить, все ли агенты почти счастливы. Если да, завершить.
    \item Выбрать агента $i$, не почти счастливого, и найти объект $j_i$:
    \[
    j_i \in \arg \max_{j=1,\ldots,n} \{a_{ij} - p_j\}.
    \]
    \item Переназначить: агент $i$ получает $j_i$, а агент, ранее назначенный на $j_i$, получает объект, принадлежавший $i$.
    \item Увеличить цену $p_{j_i}$ на:
    \[
    \gamma_i = v_i - w_i + \varepsilon,
    \]
    где $v_i = \max_j \{a_{ij} - p_j\}$, $w_i = \max_{j \neq j_i} \{a_{ij} - p_j\}$.
    \item Повторить с шага 1.
\end{enumerate}

\subsection{Сходимость алгоритма}
\begin{theorem}[Сходимость аукционного алгоритма \cite{bertsekas1990}]
\label{thm:auction_convergence}
Алгоритм завершается за конечное число шагов с распределением и ценами, почти в равновесии.
\end{theorem}

\textbf{Доказательство}. Если объект $j$ получает ставку, агент $i$, назначенный на $j$, становится почти счастливым, так как $j$ максимизирует $a_{ij} - p_j$. После увеличения $p_j$ на $\gamma_i = v_i - w_i + \varepsilon$:

\[
a_{i j} - (p_j + \gamma_i) = a_{i j} - p_j - (v_i - w_i + \varepsilon) = w_i - \varepsilon \leq \max_{k \neq j} \{a_{ik} - p_k\} - \varepsilon,
\]

что удовлетворяет условию почти счастья. Агент $i$ остается почти счастливым, пока удерживает $j$, так как цены $p_k$ ($k \neq j$) не уменьшаются, а $a_{i j} - p_j$ может только уменьшаться.

Агенты, не почти счастливые, назначены на объекты без ставок. Если все объекты получат хотя бы одну ставку, все агенты станут почти счастливы, и алгоритм завершится. Предположим, некоторые объекты никогда не получают ставок. Объект $j$ с $m$ ставками имеет цену $p_j \geq m \varepsilon$. При большом $m$ $a_{i j} - p_j$ становится меньше $a_{i k} - p_k$ для объекта $k$ без ставок ($p_k = 0$), и агент выберет $k$. Таким образом, все объекты получат ставку, или алгоритм завершится ранее, когда все агенты почти счастливы. Конечность шагов следует из конечности числа объектов.

\subsection{Оценка итераций}
\begin{claim}[Оценка итераций \cite{bertsekas1990}]
\label{claim:auction_iterations}
Количество итераций пропорционально $C / \varepsilon$, где $C = \max_{i,j} |a_{ij}|$.
\end{claim}

\textbf{Доказательство}. Каждая ставка увеличивает цену объекта минимум на $\varepsilon$. Максимальная выгода $C$ ограничивает рост цен. В худшем случае цены достигают порядка $C$, и число ставок (итераций) составляет $O(C / \varepsilon)$.

\subsection{Оптимальность}
\begin{theorem}[Оптимальность аукционного алгоритма \cite{bertsekas1990}]
\label{thm:auction_optimality}
Если алгоритм завершается с почти равновесным распределением, суммарная выгода находится в пределах $n \varepsilon$ от оптимальной. При целочисленных $a_{ij}$ и $\varepsilon < 1/n$ распределение оптимально.
\end{theorem}

\textbf{Доказательство}. Оптимальная выгода (примитивная задача):

\[
A^* = \max_{\{k_i\}} \sum_{i=1}^n a_{i k_i}, \quad k_i \neq k_m \text{ для } l \neq m.
\]

Двойственная задача:

\[
D^* = \min_{p_j} \left\{ \sum_{j=1}^n p_j + \sum_{i=1}^n \max_j \{a_{ij} - p_j\} \right\}.
\]

Для любого распределения $\{ (i, k_i) \}$ и цен $\{ p_j \}$:

\[
\sum_{i=1}^n a_{i k_i} \leq \sum_{j=1}^n p_j + \sum_{i=1}^n \max_j \{a_{ij} - p_j\},
\]

так как $\max_j \{a_{ij} - p_j\} \geq a_{i k_i} - p_{k_i}$. Следовательно, $A^* \leq D^*$.

При почти равновесии для распределения $\{ (i, j_i) \}$:

\[
a_{i j_i} - p_{j_i} \geq \max_j \{a_{ij} - p_j\} - \varepsilon.
\]

Суммируем по $i$:

\[
\sum_{i=1}^n (a_{i j_i} - p_{j_i}) \geq \sum_{i=1}^n \max_j \{a_{ij} - p_j\} - n \varepsilon.
\]

Добавим $\sum_{j=1}^n p_j$:

\[
\sum_{i=1}^n a_{i j_i} \geq \sum_{j=1}^n p_j + \sum_{i=1}^n \max_j \{a_{ij} - p_j\} - n \varepsilon.
\]

Поскольку $D^* \leq \sum_{j=1}^n p_j + \sum_{i=1}^n \max_j \{a_{ij} - p_j\}$, то:

\[
\sum_{i=1}^n a_{i j_i} \geq D^* - n \varepsilon \geq A^* - n \varepsilon.
\]

Так как $\sum_{i=1}^n a_{i j_i} \leq A^*$, получаем:

\[
A^* - n \varepsilon \leq \sum_{i=1}^n a_{i j_i} \leq A^*.
\]

Для целочисленных $a_{ij}$ и $\varepsilon < 1/n$, $A^* - \sum_{i=1}^n a_{i j_i} < 1$, и, так как разность целочисленная, $A^* = \sum_{i=1}^n a_{i j_i}$.

% Describing properties of the auction algorithm with updated details
\subsection{Свойства, преимущества и недостатки}
Свойства \cite{bertsekas1990}:
\begin{itemize}
    \item Сходимость за конечное число шагов: алгоритм завершается, как только все объекты получают хотя бы одну ставку или все люди становятся почти счастливыми (теорема \ref{thm:auction_convergence}).
    \item Оптимальность в пределах $n \varepsilon$: итоговая выгода назначения находится в пределах $n \varepsilon$ от оптимального значения, а при $\varepsilon < 1/n$ и целых $a_{ij}$ назначение оптимально (теорема \ref{thm:auction_optimality}).
    \item Производительность зависит от начальных цен: цены, близкие к оптимальным, значительно сокращают число раундов, особенно при использовании $\varepsilon$-масштабирования.
\end{itemize}

% Outlining advantages with refined insights
Преимущества:
\begin{itemize}
    \item Интуитивная экономическая модель: алгоритм имитирует реальный аукцион, упрощая понимание и интерпретацию.
    \item Гибкость для адаптации: легко модифицируется для асимметричных задач, транспортных проблем и задач минимальной стоимости потока \cite{bertsekas1990}.
    \item Хорошо подходит для параллельных и распределенных систем: допускает асинхронные и параллельные реализации, эффективен для разреженных задач \cite{gerkey2003}.
\end{itemize}

% Detailing disadvantages with precise considerations
Недостатки:
\begin{itemize}
    \item Число итераций зависит от $C / \varepsilon$: для больших $C = \max_{i,j} |a_{ij}|$ и малого $\varepsilon$ требуется больше раундов, примерно пропорционально $C / \varepsilon$.
    \item Чувствительность к выбору $\varepsilon$: большое $\varepsilon$ снижает точность (итоговая выгода дальше от оптимальной), малое $\varepsilon$ увеличивает число итераций.
    \item Чувствительность к начальным ценам: плохие начальные цены замедляют сходимость, хотя $\varepsilon$-масштабирование смягчает эту проблему.
\end{itemize}


\section{Венгерский алгоритм}
Венгерский алгоритм, разработанный Х. Куном \cite{kuhn1955}, решает задачу назначения через редукцию матрицы выгод или эквивалентную задачу максимального паросочетания в двудольном графе \cite{emaxx2025}.

\subsection{Описание через матрицу}
Алгоритм работает с матрицей выгод $\{a_{ij}\}$ размера $n \times n$ \cite{kuhn1955}:
\begin{enumerate}
    \item Для каждого ряда вычесть минимальный элемент: $a_{ij} \gets a_{ij} - \min_j a_{ij}$.
    \item Для каждого столбца вычесть минимальный элемент: $a_{ij} \gets a_{ij} - \min_i a_{ij}$.
    \item Найти минимальное число строк и столбцов, покрывающих все нули.
    \item Если число линий равно $n$, найти назначение (нули соответствуют оптимальным парам) и завершить.
    \item Иначе найти минимальный непокрытый элемент $\delta$, вычесть $\delta$ из непокрытых элементов, прибавить к элементам на пересечении линий, вернуться к шагу 3.
\end{enumerate}

\subsection{Графовый подход}
Как указано в \cite{emaxx2025}, задача назначения эквивалентна нахождению максимального паросочетания в двудольном графе $G = (V_1 \cup V_2, E)$, где $V_1$ --- агенты, $V_2$ --- объекты, а ребра $(i,j)$ имеют вес $a_{ij}$. Алгоритм:
\begin{enumerate}
    \item Построить матрицу $\{a_{ij}\}$ и редуцировать ее, как выше.
    \item Сформировать граф, где ребра соответствуют нулям в матрице.
    \item Найти максимальное паросочетание (например, с помощью алгоритма Куна или Форда-Фалкерсона).
    \item Если паросочетание покрывает $n$ вершин, оно оптимально. Иначе корректировать матрицу ($\delta$) и обновить граф.
\end{enumerate}

\subsection{Доказательство оптимальности}
Алгоритм основан на двойственности линейного программирования \cite{kuhn1955}. Примитивная задача:

\[
\max \sum_{i,j} a_{ij} x_{ij}, \quad \sum_{j} x_{ij} = 1, \sum_{i} x_{ij} = 1, x_{ij} \geq 0.
\]

Двойственная задача:

\[
\min \sum_{i} u_i + \sum_{j} v_j, \quad u_i + v_j \geq a_{ij},
\]

где $u_i$, $v_j$ --- двойственные переменные. Редукция матрицы и корректировка $\delta$ обеспечивают выполнение условий $u_i + v_j \geq a_{ij}$, а равенство $u_i + v_j = a_{ij}$ для выбранных пар дает оптимальность.

\subsection{Свойства, преимущества и недостатки}
Свойства \cite{kuhn1955, emaxx2025}:
\begin{itemize}
    \item Сходимость за конечное число шагов.
    \item Оптимальность: всегда находит максимальную выгоду.
    \item Сложность: $O(n^3)$ для матричной реализации.
\end{itemize}

Преимущества:
\begin{itemize}
    \item Гарантированная оптимальность.
    \item Простота реализации \cite{emaxx2025}.
    \item Поддержка в библиотеках.
\end{itemize}

Недостатки:
\begin{itemize}
    \item Сложность $O(n^3)$ для больших $n$.
    \item Требует полной матрицы $\{a_{ij}\}$, что проблематично при ограниченной коммуникации \cite{gerkey2003}.
    \item Нет параллелизма \cite{bertsekas1989}.
\end{itemize}

\section{Применение алгоритмов в робототехнике}
В робототехнике задача назначения используется для распределения задач между роботами в условиях ограниченной коммуникации \cite{kalyaev2009, pshikhopov2015}. Аукционный алгоритм \cite{bertsekas1990} поддерживает параллелизм, что подходит для роевых систем, но не учитывает топологические ограничения \cite{gerkey2003}. Венгерский алгоритм \cite{kuhn1955, emaxx2025} эффективен при полном доступе к данным, но неприменим без центрального управления \cite{kalyaev2009}. Исследования \cite{pshikhopov2015, gerkey2003} подчеркивают необходимость адаптации алгоритмов для динамических сетей.

\section{Постановка проблемы}
Аукционный алгоритм \cite{bertsekas1990} не учитывает ограничения связи, снижая эффективность в распределенных системах \cite{gerkey2003}. Венгерский алгоритм \cite{kuhn1955, emaxx2025} требует полной матрицы выгод, что неосуществимо без централизации \cite{kalyaev2009}. Необходим модифицированный аукционный алгоритм, который:
\begin{itemize}
    \item Учитывает ограниченные области связи.
    \item Сохраняет параллелизм.
    \item Обеспечивает эффективность и близость к оптимальности.
\end{itemize}

Цель --- разработать такой алгоритм и сравнить с венгерским методом.

\section{Выводы}
Рассмотрены аукционный \cite{bertsekas1990} и венгерский \cite{kuhn1955, emaxx2025} алгоритмы. Аукционный алгоритм поддерживает параллелизм, но требует адаптации для ограниченной коммуникации. Венгерский алгоритм оптимален, но непригоден для распределенных систем. Необходим модифицированный аукционный алгоритм, учитывающий топологические ограничения.

