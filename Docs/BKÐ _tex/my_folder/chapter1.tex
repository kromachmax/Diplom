\chapter{Анализ существующих алгоритмов назначения}
\label{ch:analysis}

\section{Введение в предметную область}
Задача назначения --- одна из ключевых проблем комбинаторной оптимизации, широко применяемая в робототехнике, логистике и управлении ресурсами. Согласно \cite{bertsekas1990}, задача заключается в распределении \( n \) агентов (например, роботов) по \( m \) объектам (например, задачам или целям) для максимизации суммарной выгоды, заданной матрицей \( \{a_{ij}\} \), где \( a_{ij} \) --- выгода от назначения агента \( i \) объекту \( j \). Математически цель задачи в общем случае:

\[
\sum_{i=1}^n a_{i j_i} \to \max,
\]

где \( j_i \) --- объект, назначенный агенту \( i \), при условии, что каждый агент получает не более одного объекта, и каждый объект назначается не более чем одному агенту. Если \( n \neq m \), некоторые агенты или объекты могут остаться неназначенными.

В робототехнике задача назначения актуальна для распределения целей между роботами в условиях ограниченной коммуникации, когда роботы обмениваются информацией только с соседями. Такие ограничения, обусловленные топологией сети, требуют алгоритмов, эффективных в распределенных системах. Примеры приложений: складская логистика, автономные транспортные системы, роевые платформы \cite{kalyaev2009, gerkey2003}.

Цель главы --- проанализировать аукционный и венгерский алгоритмы, включая их математические основы, сходимость и оптимальность, а также оценить их применимость в робототехнике. Анализ обосновывает необходимость модифицированного аукционного алгоритма для ограниченной коммуникации.


\section{Аукционный алгоритм}
Аукционный алгоритм, предложенный Д. Бертсекасом \cite{bertsekas1990}, представляет собой итеративный метод для решения задачи назначения, моделирующий процесс аукциона, где роботы делают ставки на цели, а цены корректируются для достижения оптимального распределения.

\subsection{Постановка задачи}
Задача назначения заключается в нахождении оптимального соответствия между \( n \) роботами и \( m \) целями с учетом матрицы выгод \( \{a_{ij}\} \), где \( a_{ij} \) --- выгода робота \( i \) при назначении ему цели \( j \). Цель --- максимизировать суммарную выгоду:

\[
\max \sum_{i=1}^n a_{i j_i},
\]

где \( j_i \) --- цель, назначенная роботу \( i \), при условии, что каждый робот получает не более одной цели, и каждая цель назначается не более чем одному роботу. Эта задача эквивалентна задаче линейного программирования:

\[
\begin{aligned}
&\max \sum_{i=1}^n \sum_{j=1}^m a_{ij} x_{ij}, \\
&\text{при ограничениях:} \\
&\sum_{i=1}^n x_{ij} \leq 1, \quad \forall j = 1, \ldots, m, \\
&\sum_{j=1}^m x_{ij} \leq 1, \quad \forall i = 1, \ldots, n, \\
&x_{ij} \in \{0, 1\}, \quad \forall i, j,
\end{aligned}
\]

где \( x_{ij} = 1 \), если робот \( i \) назначен цели \( j \), и \( x_{ij} = 0 \) в противном случае. Аукционный алгоритм решает эту задачу приближенно, обеспечивая суммарную выгоду, которая, согласно \cite{bertsekas1990}, находится в пределах \( \min(n, m) \varepsilon \) от оптимального значения, где \( \varepsilon > 0 \) --- параметр точности, определяющий степень приближения. Это означает, что если \( A^* \) --- оптимальная выгода, а \( A \) --- выгода, полученная алгоритмом, то:

\[
A^* - \min(n, m) \varepsilon \leq A \leq A^*.
\]

Для целочисленных \( a_{ij} \) и \( \varepsilon < 1/\min(n, m) \), алгоритм гарантирует точное оптимальное решение.

\subsection{Описание алгоритма}
Согласно \cite{bertsekas1990}, аукционный алгоритм работает с матрицей выгод \( \{a_{ij}\} \), где \( a_{ij} \) — выгода робота \( i \) при назначении цели \( j \), и использует цены \( \{p_j\} \), где \( p_j \geq 0 \) — цена, связанная с целью \( j \), отражающая текущую стоимость ее назначения и регулирующая конкуренцию между роботами. Алгоритм начинается с произвольного распределения роботов по целям (возможно, пустого) и начальных цен \( \{p_j\} \), обычно равных нулю. Ключевые определения:

\begin{itemize}
    \item \textit{Почти счастье}: Робот \( i \), назначенный цели \( j_i \), почти счастлив, если:

    \[
    a_{i j_i} - p_{j_i} \geq \max_{j=1,\ldots,m} \{a_{ij} - p_j\} - \varepsilon,
    \]

    где \( \varepsilon > 0 \) — параметр точности.
    \item \textit{Почти равновесие}: Распределение и цены, при которых все назначенные роботы почти счастливы.
\end{itemize}

Шаги алгоритма \cite{bertsekas1990}:

\begin{enumerate}
    \item Проверить, все ли назначенные роботы почти счастливы. Если да, завершить.
    \item Выбрать робота \( i \), не почти счастливого (или неназначенного), и найти цель \( j_i \):

    \[
    j_i \in \arg \max_{j=1,\ldots,m} \{a_{ij} - p_j\}.
    \]

    \item Переназначить: робот \( i \) получает цель \( j_i \), а робот, ранее назначенный на \( j_i \), получает цель, принадлежавшую \( i \), или остается неназначенным.
    \item Увеличить цену \( p_{j_i} \) на:

    \[
    \gamma_i = v_i - w_i + \varepsilon,
    \]

    где \( v_i = \max_{j=1,\ldots,m} \{a_{ij} - p_j\} \), \( w_i = \max_{j \neq j_i} \{a_{ij} - p_j\} \).
    \item Повторить с шага 1.
\end{enumerate}

\subsection{Сходимость алгоритма}
\begin{theorem}[Сходимость аукционного алгоритма \cite{bertsekas1990}]
\label{thm:auction_convergence}
Для произвольных \( n \) роботов и \( m \) целей аукционный алгоритм завершается за конечное число шагов с распределением и ценами, почти в равновесии, при котором каждый назначенный робот получает уникальную цель и является почти счастливым, а неназначенные роботы возможны только при \( n > m \).
\end{theorem}

\textbf{Доказательство}. Аукционный алгоритм итеративно назначает \( n \) роботов \( m \) целям, корректируя цены \( \{p_j\} \). Робот \( i \), назначенный цели \( j_i \), почти счастлив, если:

\[
a_{i j_i} - p_{j_i} \geq \max_{j=1,\ldots,m} \{a_{ij} - p_j\} - \varepsilon,
\]

где \( \varepsilon > 0 \) — параметр точности. На каждой итерации алгоритм выбирает робота \( i \), который либо неназначен, либо не почти счастлив, и определяет цель \( j_i \), максимизирующую:

\[
j_i \in \arg \max_{j=1,\ldots,m} \{a_{ij} - p_j\},
\]

увеличивая цену \( p_{j_i} \) на:

\[
\gamma_i = v_i - w_i + \varepsilon, \quad \text{где} \quad v_i = \max_{j=1,\ldots,m} \{a_{ij} - p_j\}, \quad w_i = \max_{j \neq j_i} \{a_{ij} - p_j\}.
\]

Робот \( i \) назначается цели \( j_i \), становясь почти счастливым, так как:

\[
a_{i j_i} - (p_{j_i} + \gamma_i) = a_{i j_i} - p_{j_i} - (v_i - w_i + \varepsilon) = w_i - \varepsilon \leq \max_{j \neq j_i} \{a_{ij} - p_j\} - \varepsilon.
\]

Если цель \( j_i \) уже была назначена роботу \( k \), то \( k \) становится неназначенным, а \( i \) получает \( j_i \). Это обеспечивает уникальность назначений: каждая цель назначается не более чем одному роботу.

Теперь объясним, почему каждый назначенный робот получает уникальную цель, и почему неназначенные роботы возможны только при \( n > m \). Алгоритм продолжает итерации, пока существуют неназначенные или не почти счастливые роботы. На каждой итерации неназначенный робот \( i \) выбирает цель \( j_i \), увеличивая ее цену на \( \gamma_i \geq \varepsilon \). Если \( j_i \) занята роботом \( k \), то \( k \) становится неназначенным и в следующей итерации выбирает новую цель. Цены \( p_j \) монотонно возрастают, что делает уже назначенные цели менее привлекательными для неназначенных роботов, так как \( a_{ij} - p_j \) уменьшается с ростом \( p_j \).

Если цель \( j \) получила много ставок, ее цена \( p_j \geq m' \varepsilon \), где \( m' \) — число ставок. При большом \( p_j \), \( a_{ij} - p_j \) становится меньше, чем \( a_{ik} - p_k \) для цели \( k \) с низкой ценой (например, \( p_k = 0 \), если \( k \) не получала ставок). Таким образом, неназначенные роботы предпочитают цели с низкими ценами, которые часто свободны или менее востребованы. Поскольку рост цен ограничен (максимум \( C + \varepsilon \), где \( C = \max_{i,j} |a_{ij}| \), так как \( a_{ij} - p_j < 0 \) невыгодно), число ставок конечно.

При \( n \leq m \), алгоритм стремится назначить каждому из \( n \) роботов уникальную цель, так как целей достаточно. Итерации продолжаются, пока все роботы не станут почти счастливыми, что возможно, так как число целей \( m \geq n \). Когда все \( n \) роботов назначены и почти счастливы, алгоритм завершается. При \( n > m \), максимум \( m \) роботов могут быть назначены, так как целей только \( m \). В этом случае после назначения \( m \) целей неназначенные \( n - m \) роботы остаются без целей, и алгоритм завершается, так как все назначенные роботы почти счастливы, а неназначенным роботам не хватает целей для новых назначений. Уникальность назначений сохраняется на всех итерациях, так как переназначение освобождает цель ровно для одного робота.

Число итераций конечно из-за конечности \( n \), \( m \) и ограниченности роста цен. Таким образом, алгоритм завершается за конечное число шагов, обеспечивая, что каждый назначенный робот получает уникальную цель и является почти счастливым, а неназначенные роботы возможны только при \( n > m \).

\subsection{Оценка итераций}
\begin{claim}[Оценка итераций \cite{bertsekas1990}]
\label{claim:auction_iterations}
Количество итераций пропорционально \( C / \varepsilon \), где \( C = \max_{i,j} |a_{ij}| \).
\end{claim}

\textbf{Доказательство}. Каждая ставка увеличивает цену цели минимум на \( \varepsilon \). Максимальная выгода \( C \) ограничивает рост цен. В худшем случае цены достигают порядка \( C \), и число ставок (итераций) составляет \( O(C / \varepsilon) \).

\subsection{Оптимальность}
\begin{theorem}[Оптимальность аукционного алгоритма \cite{bertsekas1990}]
\label{thm:auction_optimality}
Если алгоритм завершается с почти равновесным распределением, суммарная выгода находится в пределах \( \min(n, m) \varepsilon \) от оптимальной. При целочисленных \( a_{ij} \) и \( \varepsilon < 1/\min(n, m) \), распределение оптимально.
\end{theorem}

\textbf{Доказательство}. Оптимальная выгода (примитивная задача):

\[
A^* = \max_{\{k_i\}} \sum_{i=1}^n a_{i k_i}, \quad k_i \neq k_l \text{ для } l \neq i, \text{ где } k_i \in \{1, \ldots, m\}.
\]

Двойственная задача:

\[
D^* = \min_{p_j} \left\{ \sum_{j=1}^m p_j + \sum_{i=1}^n \max_j \{a_{ij} - p_j\} \right\}.
\]

Для любого распределения \( \{ (i, k_i) \} \) и цен \( \{ p_j \} \):

\[
\sum_{i=1}^n a_{i k_i} \leq \sum_{j=1}^m p_j + \sum_{i=1}^n \max_j \{a_{ij} - p_j\},
\]

так как \( \max_j \{a_{ij} - p_j\} \geq a_{i k_i} - p_{k_i} \). Следовательно, \( A^* \leq D^* \).

При почти равновесии для распределения \( \{ (i, j_i) \} \):

\[
a_{i j_i} - p_{j_i} \geq \max_j \{a_{ij} - p_j\} - \varepsilon.
\]

Суммируем по \( i \) для назначенных роботов (не более \( \min(n, m) \)):

\[
\sum_{i=1}^{\min(n, m)} (a_{i j_i} - p_{j_i}) \geq \sum_{i=1}^n \max_j \{a_{ij} - p_j\} - \min(n, m) \varepsilon.
\]

Добавим \( \sum_{j=1}^m p_j \):

\[
\sum_{i=1}^{\min(n, m)} a_{i j_i} \geq \sum_{j=1}^m p_j + \sum_{i=1}^n \max_j \{a_{ij} - p_j\} - \min(n, m) \varepsilon.
\]

Поскольку \( D^* \leq \sum_{j=1}^m p_j + \sum_{i=1}^n \max_j \{a_{ij} - p_j\} \), то:

\[
\sum_{i=1}^{\min(n, m)} a_{i j_i} \geq D^* - \min(n, m) \varepsilon \geq A^* - \min(n, m) \varepsilon.
\]

Так как \( \sum_{i=1}^{\min(n, m)} a_{i j_i} \leq A^* \), получаем:

\[
A^* - \min(n, m) \varepsilon \leq \sum_{i=1}^{\min(n, m)} a_{i j_i} \leq A^*.
\]

Для целочисленных \( a_{ij} \) и \( \varepsilon < 1/\min(n, m) \), \( A^* - \sum_{i=1}^{\min(n, m)} a_{i j_i} < 1 \), и, так как разность целочисленная, \( A^* = \sum_{i=1}^{\min(n, m)} a_{i j_i} \).

\subsection{Преимущества и недостатки}
Преимущества:
\begin{itemize}
    \item Интуитивная экономическая модель: алгоритм имитирует реальный аукцион, упрощая понимание и интерпретацию.
    \item Гибкость для адаптации: легко модифицируется для асимметричных задач с \( n \neq m \), транспортных проблем и задач минимальной стоимости потока \cite{bertsekas1990}.
    \item Хорошо подходит для параллельных и распределенных систем: допускает асинхронные и параллельные реализации, эффективен для разреженных задач \cite{gerkey2003}.
\end{itemize}

Недостатки:
\begin{itemize}
    \item Число итераций зависит от \( C / \varepsilon \): для больших \( C = \max_{i,j} |a_{ij}| \) и малого \( \varepsilon \) требуется больше итераций, примерно пропорционально \( C / \varepsilon \).
    \item Чувствительность к выбору \( \varepsilon \): большое \( \varepsilon \) снижает точность, малое \( \varepsilon \) увеличивает число итераций.
    \item Чувствительность к начальным ценам: плохие начальные цены замедляют сходимость, хотя \( \varepsilon \)-масштабирование смягчает эту проблему.
\end{itemize}

\section{Венгерский алгоритм}
Венгерский алгоритм, разработанный Х. Куном \cite{kuhn1955}, решает задачу назначения через редукцию матрицы выгод или эквивалентную задачу максимального паросочетания в двудольном графе \cite{emaxx2025}.

\subsection{Матричный подход}
Алгоритм работает с матрицей выгод \( \{a_{ij}\} \) размера \( n \times m \). Для унификации, если \( n \neq m \), матрица дополняется фиктивными роботами или целями с нулевой выгодой, чтобы получить квадратную матрицу размера \( \max(n, m) \times \max(n, m) \) \cite{kuhn1955}:

\begin{enumerate}
    \item Для каждого ряда вычесть минимальный элемент: \( a_{ij} \gets a_{ij} - \min_j a_{ij} \).
    \item Для каждого столбца вычесть минимальный элемент: \( a_{ij} \gets a_{ij} - \min_i a_{ij} \).
    \item Найти минимальное число строк и столбцов, покрывающих все нули.
    \item Если число линий равно \( \min(n, m) \), найти назначение (нули соответствуют оптимальным парам) и завершить.
    \item Иначе найти минимальный непокрытый элемент \( \delta \), вычесть \( \delta \) из непокрытых элементов, прибавить к элементам на пересечении линий, вернуться к шагу 3.
\end{enumerate}

\subsection{Графовый подход}
Как указано в \cite{emaxx2025}, задача назначения эквивалентна нахождению максимального паросочетания в двудольном графе \( G = (V_1 \cup V_2, E) \), где \( V_1 \) --- роботы (\( n \) вершин), \( V_2 \) --- цели (\( m \) вершин), а ребра \( (i,j) \) имеют вес \( a_{ij} \). Алгоритм:

\begin{enumerate}
    \item Построить матрицу \( \{a_{ij}\} \) и редуцировать ее, как выше.
    \item Сформировать граф, где ребра соответствуют нулям в матрице.
    \item Найти максимальное паросочетание (например, с помощью алгоритма Куна или Форда-Фалкерсона).
    \item Если паросочетание покрывает \( \min(n, m) \) вершин, оно оптимально. Иначе корректировать матрицу (\( \delta \)) и обновить граф.
\end{enumerate}

\subsection{Доказательство оптимальности}
Алгоритм основан на двойственности линейного программирования \cite{kuhn1955}. Примитивная задача:

\[
\max \sum_{i,j} a_{ij} x_{ij}, \quad \sum_{j=1}^m x_{ij} \leq 1, \sum_{i=1}^n x_{ij} \leq 1, x_{ij} \geq 0.
\]

Двойственная задача:

\[
\min \sum_{i=1}^n u_i + \sum_{j=1}^m v_j, \quad u_i + v_j \geq a_{ij},
\]

где \( u_i \), \( v_j \) --- двойственные переменные. Редукция матрицы и корректировка \( \delta \) обеспечивают выполнение условий \( u_i + v_j \geq a_{ij} \), а равенство \( u_i + v_j = a_{ij} \) для выбранных пар дает оптимальность.

\subsection{Преимущества и недостатки}
Преимущества:
\begin{itemize}
    \item Гарантированная оптимальность.
    \item Простота реализации \cite{emaxx2025}.
    \item Поддержка в библиотеках.
\end{itemize}

Недостатки:
\begin{itemize}
    \item Сложность \( O(n^2*m) \) для больших \( n \) или \( m \).
    \item Требует полной матрицы \( \{a_{ij}\} \), что проблематично при ограниченной коммуникации \cite{gerkey2003}.
    \item Непригодность для распределенных систем: невозможность работы в условиях ограниченной коммуникации между роботами, так как алгоритм предполагает централизованное управление \cite{kalyaev2009}.
    \item Нет параллелизма \cite{bertsekas1989}.
\end{itemize}

\section{Постановка проблемы}
В распределенных системах с \( n \) роботами и \( m \) целями часто возникают ограниченные области связи, при которых каждый робот имеет доступ к полной информации о матрице выгод \( \{a_{ij}\} \), то есть видит все цели, но не может обмениваться информацией с другими роботами \cite{gerkey2003, kalyaev2009}. Аукционный алгоритм \cite{bertsekas1990} предполагает, что роботы могут координировать назначения через обмен информацией, что невозможно в условиях отсутствия коммуникации между роботами. Это приводит к необходимости дополнительных механизмов координации или снижению качества назначений. Венгерский алгоритм \cite{kuhn1955, emaxx2025} требует полной матрицы выгод и централизованного управления, что неосуществимо в таких системах \cite{kalyaev2009}. Необходим модифицированный аукционный алгоритм, который:

\begin{itemize}
    \item Учитывает отсутствие коммуникации между \( n \) роботами при доступе к полной информации о \( m \) целями.
    \item Обеспечивает эффективность и близость к оптимальности.
\end{itemize}

Цель --- разработать такой алгоритм и сравнить его с венгерским методом.

\section{Выводы}
Рассмотрены аукционный \cite{bertsekas1990} и венгерский \cite{kuhn1955, emaxx2025} алгоритмы для задачи назначения \( n \) роботов по \( m \) целям. Аукционный алгоритм эффективен, но требует адаптации для условий, где роботы не могут обмениваться информацией друг с другом, несмотря на доступ к полной матрице выгод. Венгерский алгоритм оптимален, но непригоден для распределенных систем без централизации. Необходим модифицированный аукционный алгоритм, учитывающий отсутствие коммуникации между роботами.


