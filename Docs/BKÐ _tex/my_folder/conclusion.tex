\chapter*{Заключение} \label{ch-conclusion}
\addcontentsline{toc}{chapter}{Заключение}	% в оглавление 


Проведенное исследование демонстрирует, что разработанный модифицированный аукционный алгоритм является эффективным решением для задач оптимального распределения целей в распределенных системах управления роботами, функционирующих в условиях ограниченных областей связи. В отличие от классического венгерского алгоритма, который требует полной матрицы выгод и централизованного управления, что делает его непригодным для децентрализованных систем, модифицированный аукционный алгоритм учитывает топологические ограничения сети, позволяя роботам обмениваться информацией только внутри связных компонент. Это обеспечивает его применимость в реальных сценариях, где коммуникация между роботами ограничена радиусом связи.

Экспериментальные результаты, представленные в главе \ref{ch4}, подтверждают высокую эффективность алгоритма: он достигает высокой точности при разумных значениях радиуса связи и параметра $\varepsilon$, сохраняя приемлемое время выполнения. Алгоритм требует меньшего числа арифметических и логических операций по сравнению с венгерским, что делает его предпочтительным при высоких скоростях обмена данными ($F_{\text{обм}}$), несмотря на большее число итераций.

Модифицированный аукционный алгоритм обладает значительным потенциалом для применения в таких областях, как складская логистика, автономные транспортные системы и роевые робототехнические платформы. Однако отсутствие глобальной оптимальности из-за ограниченной коммуникации между компонентами указывает на необходимость дальнейших исследований, включая разработку механизмов ограниченного обмена информацией и адаптивной настройки параметра $\varepsilon$. Перспективы, описанные в главе  \ref{ch5}, подчеркивают возможность интеграции алгоритма в динамические сети и повышения его робастности к погрешностям измерений, что делает его универсальным инструментом для современных робототехнических систем.
