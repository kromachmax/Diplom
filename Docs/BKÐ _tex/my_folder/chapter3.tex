\chapter{Реализация и программное обеспечение} \label{ch3}

Глава посвящена реализации модифицированного аукционного алгоритма и венгерского алгоритма для сравнительного анализа. Описывается программная среда, выбранная для реализации, включая язык программирования и инструменты моделирования распределенных систем роботов. Приводится структура программного обеспечения, обеспечивающего тестирование алгоритмов в условиях ограниченной коммуникации. Рассматриваются аспекты оптимизации кода для параллельных вычислений и интеграции с робототехническими платформами.

\section{Программная среда}

Реализация выполнена на языке C++ с использованием библиотек Qt для визуализации данных. Генерация случайных данных реализована с использованием стандартных средств C++. Код совместим с платформами Windows и Unix.

\section{Структура реализации}

Программное обеспечение включает модули для аукционного и венгерского алгоритмов, а также логику тестирования и визуализации результатов. Аукционный алгоритм учитывает ограничения коммуникации, разделяя роботов на группы по связности и назначая задачи с учетом настраиваемого параметра радиуса связи, влияющего на скорость и точность. Венгерский алгоритм решает задачу оптимального назначения. Тестирование проводится на случайных данных с варьированием размеров задачи (от 5 до 300 роботов/целей), радиусов связи (от 1 до 300 условных единиц) и параметра $\varepsilon$ алгоритма (от $10^{-6}$ до 10.0).

\section{Оптимизация}

Для повышения производительности применены:
\begin{itemize}
  \item Параллельные вычисления для обработки независимых групп роботов в аукционном алгоритме.
  \item Эффективные структуры данных для сокращения затрат памяти и времени.
  \item Настройка визуализации для наглядного представления результатов.
\end{itemize}

\section{Интеграция с робототехническими системами}

Реализация может быть адаптирована для робототехнических платформ, таких как ROS, с формированием данных на основе сенсоров. Возможна разработка веб-интерфейса для визуализации распределения задач. Для реальных систем потребуется обработка динамических изменений и сбоев связи.

\section{Тестирование и визуализация}

Тестирование включало сравнение времени выполнения и точности алгоритмов. Результаты представлены графиками, созданными с использованием Qt, что позволило оценить влияние параметров на производительность.

\section{Выводы}

Разработанное программное обеспечение обеспечивает сравнение аукционного и венгерского алгоритмов в условиях ограниченной коммуникации. Реализация в C++ с применением параллельных вычислений демонстрирует высокую производительность и готовность к адаптации для реальных систем.