\documentclass{beamer}

% Setting up encoding and language support
\usepackage[T2A]{fontenc}
\usepackage[utf8]{inputenc}
\usepackage[russian]{babel}

% Including necessary packages for tables, algorithms, and math
\usepackage{array}
\usepackage{booktabs}
\usepackage{algorithmicx}
\usepackage{algpseudocode}
\usepackage{amsmath}
\usepackage{graphicx}

% Configuring listings for code display
\usepackage{listings}
\lstset{
	basicstyle=\ttfamily\footnotesize,
	keywordstyle=\color{blue},
	commentstyle=\color{green!50!black},
	stringstyle=\color{red},
	showstringspaces=false,
	breaklines=true,
	frame=single
}

% Setting Beamer theme and color scheme
\usetheme{Madrid}
\usecolortheme{dolphin}

% Customizing Beamer colors
\setbeamercolor*{palette primary}{use=structure,fg=white,bg=structure.fg!60!blue}
\setbeamercolor*{palette secondary}{use=structure,fg=white,bg=structure.fg!45!blue}
\setbeamercolor*{palette tertiary}{use=structure,fg=white,bg=structure.fg!30!blue}
\setbeamercolor*{alerted text}{fg=blue!80!cyan}
\setbeamercolor*{block title}{bg=blue!15!white,fg=blue!80!black}
\setbeamercolor*{block body}{bg=blue!10!white}
\setbeamercolor*{block title alerted}{bg=blue!20!white,fg=blue!80!black}
\setbeamercolor*{block body alerted}{bg=blue!15!white}
\setbeamercolor{background canvas}{bg=blue!10}

% Defining the title page layout
\title{Модифицированный аукционный алгоритм}
\subtitle{Децентрализованное назначение целей группе роботов}
\author{\hspace*{\fill}Кромачев Максим\\\hspace*{\fill}\footnotesize СПбПУ Петра Великого}
\date{\hspace*{\fill}\today}

% Кастомизация титульной страницы
\setbeamertemplate{title page}{
	\vbox{}
	\vfill
	\begingroup
	\begin{beamercolorbox}[sep=0.1pt,wd=\textwidth,center]{title}
		\usebeamerfont{title}\inserttitle\par
	\end{beamercolorbox}
	\vskip0.1em
	\begin{beamercolorbox}[sep=8pt,wd=\textwidth,center]{subtitle}
		\usebeamerfont{subtitle}\insertsubtitle\par
	\end{beamercolorbox}
	\vskip1.5em
	\begin{beamercolorbox}[sep=8pt,wd=\textwidth,right]{author}
		\usebeamerfont{author}\insertauthor
	\end{beamercolorbox}
	\vskip0.5em
	\begin{beamercolorbox}[sep=8pt,wd=\textwidth,right]{date}
		\usebeamerfont{date}\insertdate
	\end{beamercolorbox}
	\endgroup
	\vfill
}

% Adding frame number to footline
\setbeamertemplate{footline}[frame number]

\begin{document}
	
	\begin{frame}
		\titlepage
	\end{frame}
	
	\begin{frame}
		\frametitle{Постановка задачи}
		\framesubtitle{Назначение целей группе роботов}
		
		\begin{block}{Общая задача}
			\begin{itemize}
				\item Задача назначения — ключевая проблема комбинаторной оптимизации в робототехнике, логистике, управлении ресурсами.
				\item Распределение $n$ роботов по $m$ целям для максимизации суммарной выгоды:
				\[
				\max \sum_{i=1}^n a_{ij_i}, \quad \text{где } a_{ij} \text{ — выгода назначения робота } i \text{ цели } j.
				\]
				\item Ограничения:
				\[
				\begin{aligned}
					&\sum_{i=1}^n x_{ij} \leq 1, \quad \forall j = 1, \ldots, m, \\
					&\sum_{j=1}^m x_{ij} \leq 1, \quad \forall i = 1, \ldots, n, \\
					&x_{ij} \in \{0, 1\}, \quad \forall i, j,
				\end{aligned}
				\]
				где $x_{ij} = 1$, если робот $i$ назначен цели $j$.
			\end{itemize}
		\end{block}
	\end{frame}
	
	\begin{frame}
		\frametitle{Алгоритм аукциона без модификации}
	\end{frame}

	\begin{frame}
	\frametitle{Описание аукционного алгоритма}
	\framesubtitle{Шаги и ключевые понятия}
	\end{frame}
	
	\begin{frame}
		\frametitle{Венгерский алгоритм}
	\end{frame}

	\begin{frame}
		\frametitle{Описание венгерского алгоритма}
		\framesubtitle{Шаги и ключевые понятия}
	\end{frame}

	\begin{frame}
		\frametitle{Преимущества и недостатки методов}
		\framesubtitle{Сравнение аукционного и венгерского методов}
	\end{frame}
	
	
	\begin{frame}
		\frametitle{Модификация аукционного алгоритма}
		\framesubtitle{Учёт ограничений связи}
	\end{frame}
	
	\begin{frame}
		\frametitle{Анализ модифицированного аукционного алгоритма}
		\framesubtitle{Сравнение с классическими методами}
	\end{frame}
	
	
	\begin{frame}
		\frametitle{Влияние параметров}
		\framesubtitle{Ключевые зависимости}
	\end{frame}
	
	\begin{frame}
		\frametitle{Практическая значимость}
	\end{frame}
	
	\begin{frame}
		\frametitle{Перспективы развития}
	\end{frame}
	
	\begin{frame}
		\frametitle{Выводы}
	\end{frame}
	
\end{document}